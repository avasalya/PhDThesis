% ======================================================================= %
%                        Distinct Motor Contagions
% ======================================================================= %

{\color{blue}\chapter{Distinct motor contagions}\label{distinct motor contagion}}

\section{Introduction}

Several studies in the past have demonstrated that just by observing an action performed by human or robot can affect the movements of an observing human; an effect widely known as motor contagion. Although, these previous studies have either analyzed the motor contagions induced during (which we call \emph{on-line} contagions), or induced after (\emph{off-line} contagions) the observation of the robot, but never together. Therefore, it is still unclear whether and how these two motor contagions differ from each other. In this Chapter, we designed a paradigm inspired by the industrial co-worker setting and examined the differences between the induced \textit{on-line} and \textit{off-line} contagions in participants by the observation of the same movements performed by both human and humanoid robot co-worker. We specifically examined three questions:

\begin{enumerate}
	\item Can on-line and off-line contagions from the observation of a same movement affect different movement features of the human participant?
	\item how do the strengths of the on-line and off-line contagions vary with the nature of the co-worker (ie. if human or robot), and the behavior of the co-worker?
	\item Consequently, are the on-line and off-line contagions different, or do they constitute the same effect observed at different instances?
\end{enumerate}


%%%%%%%%%%%%%%%%%%%%%%%%%%%%%%%%%%%%%%%%%%%%%%%%%%%%%%%%%%%%%%%%%%%%%%%%%%%%%%%%
%                                                                              %
%		                           M E T H O D S                               %
%                                                                              %
%%%%%%%%%%%%%%%%%%%%%%%%%%%%%%%%%%%%%%%%%%%%%%%%%%%%%%%%%%%%%%%%%%%%%%%%%%%%%%%%
%\clearpage

\section{Materials and methods}\label{roman method}

\subsection{Participants}

In total 45 healthy adults participated in our study. 3 participants (2 males and a female of 3 nationalities, 29.6$\pm$5, mean$\pm$SD, aged 25-35) worked as volunteer models for capturing the human arm motion data. A total of 42 participants (22 females, 20 males of 11 nationalities, age between 20-39, mean$\pm$SD, 25.9$\pm$4.35) participated as `co-worker' in the main study. All human participants had normal or corrected to normal vision. However according to the {\it Edinburgh Handedness Inventory}~\cite{robinson2013edinburgh}, three of them were left-handed. We had received prior approval from the local ethics committee at the National Institute of Advanced Industrial Science and Technology (AIST) in Tsukuba, Japan to conduct these experiments.

Before beginning the experiment, all participants were carefully instructed and informed regarding the experiment and task procedure. All of the participants agreed and gave their written consent to participate in this study. However, participants were unaware of the motives of the experiment as they were not told what aspect of their behavior we will be analyzing later. This na\"ive nature of the participants were crucial as to avoid biasing in the results, since we were mainly interested in the implicit effects of motor contagions. Each participant was awarded with 2021 Japanese Yen for their valuable contribution.

Participants for the study were recruited through an advertisement via a local event forum, Facebook page of our experiment and via word of mouth in the Tsukuba University, Tsukuba, Japan. There were no restrictions, apart from participants needed to be at least 18 years old to participate in this experiment.


\begin{figure}[hb]
	\centering{\includegraphics[width=1\columnwidth]{plots/c2-plots/setup}}
	\caption{Experimental setup: The participants in our experiment worked in three conditions; (i) with a robot co-worker performing \textit{biological} movements ($\textit{R}_{\text{biol}}$), (ii) a human co-worker ($\textit{H}$), and (iii) a robot co-worker performing \textit{non-biological} movements ($\textit{R}_{\text{nonbiol}}$). The coordinate axis defining the movement setup is indicated in white and fixed on the participant's table.}
	\label{fig:setup}
\end{figure}


\subsection{Setup}

The experimental setup is shown in Fig.~\ref{fig:setup}. Initially participants were requested to sit comfortably on a chair in front of a large table. While the `co-worker' was sitting on the other side of the same table. The `co-worker' was either a humanoid robot or a human experimenter. A touchscreen was placed horizontally underneath the table such that both the co-worker and the participant were presented with a unique pair of red circles on their respective side of the table. Each circle for participants had a diameter $\oslash$5cm and the circles for co-worker had a diameter $\oslash$9cm, also each pair of circles were at a distance of 50cm from each other. We had enclosed the experimental setup by movable panels and covered the panel behind the co-worker with a dark grey curtain. 

We placed ten passive reflective markers on the hands, arms, elbows and shoulders of the participants and co-worker. We used six kestrel infra-red cameras (\texttt{Motion Analysis Co.,}) at 200Hz to track the position of these markers.

A biped humanoid robot HRP-2Kai (154cm tall, 58kg, 32DOF) was used as robot co-worker~\cite{Kaneko:RAS_ICHR:2015}. A male, trained experimenter (age: 37) was assigned to act as the human co-worker. During the experiment both co-workers used their right hand.


\begin{figure*}[phtb]
	\includegraphics[width=\textwidth,height=6cm]{plots/c2-plots/trialprotocol}
	\caption{A) Trial protocol: The participants worked in repeated trials with either a robot or human co-worker (the figure shows the trial with a human co-worker). Each trial consisted of a period when the participant worked alone and co-worker relaxed (participant-alone period), both worked together (together period), and the co-worker worked alone (co-worker-alone period). The notation of the kinematic and time variables (represented in general by \boldmath{$\eta$}) in each period are shown in the figure. B) The trajectories made by the robot co-worker in the $\textit{R}_{\text{biol}}$ and $\textit{R}_{\text{nonbiol}}$ conditions. C) The time trajectories followed by the robot co-worker in the $\textit{R}_{\text{nonbiol}}$ condition in the Y and Z dimension, and the via-points (blue circles) used to generate the trajectory.}
	\label{fig:trial}
\end{figure*}

\subsection{Experimental task and conditions}

The task to carry out in our experiment was inspired from the industrial \textit{pick-n-place} part-assembly task. Participants were required to touch the red circles repeatedly on the touchscreen with a stylus in their right hand. The same task was performed by the co-worker (human or the HRP-2Kai robot) in front of the participants. The participants worked in a series of 50 second \textit{trials} with the co-worker. In each trial, they initially performed alone for 10 seconds (participant-alone period), performed with the co-worker for the next 20 seconds (together period), and then relaxed while watching the co-worker performs its/his task for the last 20 seconds (co-worker-alone period) (Fig.~\ref{fig:trial}A).

We instructed the participants to ``{\it always hold the stylus like a stamp and touch alternatively inside each red circle on the touchscreen with continuous and smooth hand movements at a comfortable speed}''. Participants were not instructed regarding the movement or speed of their hand trajectories. All participants wore headphones (through which white noise was sent) and had no audio feedback of the noise from the moving robot (confirmed in the post experiment questionnaire). They were specifically told to ``{\it focus on your own task and ignore the co-worker when he/it starts after them}''.

In total, participants behavior was tested in six experimental conditions. In each condition, co-worker's behavior such as physical appearance or movement trajectory was changed. In this chapter we only discuss our findings and present results from three conditions relevant for distinguishing the \textit{on-line} and \textit{off-line} contagions. Whereas in the next Chapter (\nameref{more than just co-workers}), we have discussed the effect of physical features of the co-worker on the participant's behavior. 

At the beginning, the participants worked with a human co-worker in the Human ($H$) condition. The HRP-2Kai robot acted as the robot co-worker in the Robot biological ($R_{\text{biol}}$) condition and played back (biological) hand movements. These biological movements of a human volunteer (blue plot in Fig.~\ref{fig:trial}B), were recorded in a preliminary experiment (also see section~\ref{hrpTraj}). Finally, the participants worked again with the HRP-2Kai robot as the co-worker, in the Robot non-biological ($R_{\text{nonbiol}}$) condition, where the robot now performed a non-biological movement profile. This profile was roughly trapezoidal in shape and velocity profile (magenta plot in Fig.~\ref{fig:trial}B). 

Each participant worked in maximum three conditions. Our experiment consists of six \emph{condition combination groups}, such that each participant in a combination group worked in the $R_{\text{biol}}$ condition, and two out of five remaining conditions. This provided us with the opportunity to compare participant's behavior in any condition and against his/her behavior in the $R_{\text{biol}}$ condition. Since each of our experimental condition lasted over 20 minutes, which made the total time of experiment over an hour. Therefore to avoid participants being tired, we didn't allow them to experience all the conditions. Note that the order of conditions was chosen randomly across the participants. In this Chapter we discuss our results from the participants in $R_{\text{nonbiol}}$ and/or $H$ conditions, in addition to the behavior of the same participant's $R_{\text{biol}}$ condition.

The participant worked in a series of 10 trials in each condition. The co-worker performed at a pseudo-randomly selected constant yet unique frequency (in the range of 0.16 to 1.1Hz) in each trial. This pseudo-random nature of the co-worker performance was critical to avoid contamination by behavioral drifts across trials. A metronome using headphones were provided to the human co-worker (like in~\cite{Bisio:PlosOne:2014}), to cue him/her of the required movement frequency, and assisting to keep and maintain the particular movement frequency.

\subsection{HRP-2Kai movement trajectories} \label{hrpTraj}

We played back recorded human hand movement as the arm movements of HRP-2Kai in the $R_{\text{biol}}$ condition. In a preliminary experiment, we recorded the hand movements of three volunteers (a female and 2 males). {\it Motion Analysis Co.,} motion tracking system was used to record their hand movements, while at the same time, volunteers were cued by an audio metronome to help maintain their hand movement frequency. These movements were acquired at several frequencies between 0.16 to 1.1Hz. Interestingly these three volunteers movements were found statistically similar in the $x$, $y$ and $z$ velocity profiles ($p > 0.05$), and showed similar fashion in movement height with movement frequency, \texttt{i.e.}, trajectory height consistently decreased with increase of movement frequency. Therefore in this experiment, we chose to only use the recorded movements from one of the male volunteer. In order to maintain the trajectory shape and also the characteristic variance of human trajectories, we purposely decided that it is better than taking an average of trajectories by the three volunteers. 

In our task, the human movements were distinguished by smooth velocity changes but did not exhibit any direction changes (via points). Therefore for the $R_{\text{nonbiol}}$ condition, we designed a via-points based robot trajectory. Inspired by the industrial manipulators trajectories of constant velocity phase and trapezoidal shape during {\it pick-n-place} task and keeping in mind our HRP-2Kai joint constraints during fast movements, we redesigned a trapezoidal shape trajectory for this condition with smooth curves between the acceleration and deceleration transition phases. Our task required the humans to make movements mainly in the YZ plane, hence we also designed the robot arm trajectory in the YZ plane. Using two temporal via-points~\cite{Biagiotti:Springer:2008}, we designed a smooth piece-wise polynomial trajectory in position-time using the third order polynomial segments. The boundary conditions restricted the slope (velocity) to zero at the start, the end and the via-points (see Fig.~\ref{fig:trial}C). The initial ($y_0$) and final ($y_f$) positions in \texttt{Y} were set to zero and 50cm respectively, as per the movements required by the participants. To give a non-biological behavior, we set the maximum \texttt{Z} height ($z_{\max}$) for the robot in the $R_{\text{nonbiol}}$ condition to 13cm one way and 8cm the other. 


%%%%%%%%%%%%%%%%%%%%%%%%%%%%%%%%%%%%%%%%%%%%%%%%%%%%%%%%%%%%%%%%%%%%%%%%%%%%%%%%
%                                                                              %
%		                         D A T A    A N A L Y S I S                    %
%                                                                              %
%%%%%%%%%%%%%%%%%%%%%%%%%%%%%%%%%%%%%%%%%%%%%%%%%%%%%%%%%%%%%%%%%%%%%%%%%%%%%%%%

%\clearpage

\section{Data analysis}

\subsection{Variables}

Our analysis is based on the position data of the markers placed on the participant's and co-worker's stylus. In order to bring out the possible behavioral differences between the movements towards and back between the touches on the touchscreen, we analyzed behavioral variables across each movement between the red circles on the touchscreen, which we call as \textit{iterations} (such that two consecutive iterations constituted a movement cycle). The participants and co-workers made non-stop continuous movements between the touches, and hence we could extract individual iterations by a participant or co-worker by examining the directional changes of their $y$-velocity in the recorded motion capture data. In this Chapter, we focused our efforts on the kinematic variables along the \texttt{Y} and \texttt{Z} axes inside each iteration and analyzed the maximum movement length ($y_{\max}$), maximum movement height ($z_{\max}$), maximum absolute velocities ($\max|\dot{y}|$, $\max|\dot{z}|$), mean absolute velocities ($|\overline{\dot{y}}|$, $|\overline{\dot{z}}|$), maximum accelerations ($\max(\ddot{y})$, $\max(\ddot{z})$), and minimum accelerations ($\min(\ddot{y})$, $\min(\ddot{z})$), by the participants and co-workers to understand whether and how the co-worker behavior were affected by the \textit{on-line} and \textit{off-line} contagions. In addition to the kinematic variables, we also analyzed the time between the touches in each iteration, which we will refer to as the half-time period or ({\it htp}).

\subsection{Participant sample size}\label{roman sample size}

We initially conducted our experiment with the sample size of 35 participants such that all of them participated in the $R_{\text{biol}}$ condition and 14 participants in each of the five other conditions. We call it as \textit{participant group}, where participants performed in $R_{\text{biol}}$ condition along with one of the five other condition. Thus a total of five \textit{participant group}s. Note that these groups are different from the previously mentioned six \emph{condition combination group}s. The number `14' also represents the participant numbers in similar previous studies~\cite{Bisio:PlosOne:2010, Bisio:PlosOne:2014} and this participant number `14' also corresponds to the G* power analysis~\cite{Erdfelder:JBRMIC:1996} using two-way one sample {\it T}-test ($\alpha$ = 0.05, $\beta$ = 0.85, $d$ = 0.9)~\cite{Verma:power_analysis:2017} for the biological experiments. We discovered notable motor contagions in the {\it htp}s in the $R_{\text{biol}}$ condition (median = 0.014, Z(31) = 3.14, $p = 0.0016$, 3 participants with slopes beyond the 95\% confidence interval were removed as outliers). We thus believed that a positive {\it htp} slope in $R_{\text{biol}}$ condition as true, and checked the {\it htp} values in the $R_{\text{biol}}$ conditions in each participant group. But with these participants numbers, the {\it htp} slopes in the $R_{\text{biol}}$ condition were not significant across the participant groups ($p<0.05$, one-way ANOVA). The {\it htp} slope during $R_{\text{biol}}$ condition were observed to be significant with two participant groups ($p = 0.022$, $p = 0.038$), marginally significant with other two participant groups ($p = 0.07$, $p = 0.08$) and not significant in one participant group ($p = 0.36$). Hence, we later proposed and decided to add 7 participants (50\%) across these groups, making a total of 42 participants. This addition of participants ensured that the {\it htp} slopes across the participant groups become similar ($P = 0.99$; one-way Kruskal-Wallis H-test). After removal of three outliers, this gave us participants numbers of 13 in the $H$ condition, 18 in the $R_{\text{nonbiol}}$, and 39 in total for the $R_{\text{biol}}$ condition.


\begin{figure}[t]
	\centering{\includegraphics[width=1\columnwidth]{plots/c2-plots/all_fit}}
	\caption{Examples of linear regression fits obtained in the $\textit{H}$ (orange), $\textit{R}_{\text{biol}}$ (blue), $\textit{R}_{\text{nonbiol}}$ (magenta) conditions: A) \textit{off-line} contagions in the participant's \boldmath{$|\overline{\dot{y}}|$} ($\textbf{y}$-axis) as a function of co-worker's \boldmath{$|\overline{\dot{y}}|$} ($\textbf{x}$-axis); B) \textit{On-line} contagions in the participant's \textit{htp}s ($\textbf{y}$-axis) as a function of co-worker's \textit{htp}s ($\textbf{x}$-axis). We used the AIC to choose either a first or second order model to fit the data for each participant. The lines represent the tangent slopes at the minimal co-worker feature value.}
	\label{fig:fitting}
\end{figure}



\subsection{Quantifying the \textit{off-line} contagions}

After observing the co-worker, we quantified the participant's change in behavior by analyzing how the average value of a given kinematic or time variable $\eta_p$ by a participant during the participant-alone period in trial $i$ ($\eta_p^a (i)$) has changed, compared with that of the co-worker in the co-worker-alone period of the previous trial ($\eta_c(i-1)$). We later used first order or second order regression model to explain the data and performed the regression using MATLAB's  \texttt{fitlm} function. The first or second order regression models were chosen based on the Akaike Information Criteria, or AIC~\cite{Akaike:ISIT:1973}. Some examples of fittings are illustrated in Fig.~\ref{fig:fitting}A. We then gathered the slope at the minimum co-worker variable value ($\min[\eta_c(i)]$) across participants. The gathered slope data for each variable and condition was then checked for normality using the {\it Shapiro-Wilk} test and further analyzed for a difference from zero either using a one-sample {\it T}-test or a {\it Signed Rank} test based on whether the distribution was normal or not, respectively. Fig.~\ref{fig:offline}A illustrate the data plots of $|\overline{\dot{y}}|$ from the three reported conditions.

\subsection{Quantifying the \textit{on-line} contagions}

To quantify the effects due to the \textit{on-line} contagions, we now looked again at the average value of each of the analyzed kinematic or time variable $\eta_p$ but this time in the together period. Note that in order to remove any persistent \textit{off-line} contagions in this period, we regressed the \textit{change} in the participant's behavior, between the together period and alone-period in a trial ($\eta_p^t (i)- \eta_p^a (i)$), and the corresponding value of the same variable in the co-worker behavior in the same trial $\eta_c(i)$. A first order, or second order regression model was chosen again using AIC for each participant, and similarly with the \textit{off-line} contagion analysis, the slope of tangent at the minimum co-worker variable value ($\min[\eta_c(i)]$) was gathered across participants, then checked for normality using the {\it Shapiro-Wilk}, and finally analyzed for difference from zero using a one-sample {\it T}-test or a {\it Signed Rank} test. The Fig.~\ref{fig:fitting}B illustrates the fitting of {\it htp} in representative participants in the three reported conditions and the collection of slopes are in shown in Fig.~\ref{fig:online}B.

\begin{figure}[t]
	\centering{\includegraphics[width=0.8\columnwidth]{plots/c2-plots/offline_result_vel_htp}}	
	\caption{The \textit{off-line} contagions: Observed changes in the participant's \boldmath{$|\overline{\dot{y}}|$} and \textit{htp} in the $\textit{H}$ (orange plots), $\textit{R}_{\text{biol}}$ (blue plots), $\textit{R}_{\text{nonbiol}}$ (magenta plots) conditions. All p values are Bonferroni corrected.}
	\label{fig:offline}
\end{figure}

\subsection{Statistical correction}

As mentioned earlier, every participant in this study performed in three conditions: the $R_{\text{biol}}$ condition, and two of the remaining five conditions. therefore we make two comparisons for each participant, between $R_{\text{biol}}$ and the two other conditions. Correspondingly, in our comparisons in Fig.~\ref{fig:online}, we use a Bonferroni correction of (3 conditions $-$ 1) 2, and hence all $p$ values below $0.05$ were multiplied by 2.

\subsection{Movement congruency analysis} \label{congSec}

Moreover, during the \textit{on-line} contagions, we also investigated if movement congruency between the participant and the co-worker influenced the \textit{on-line} contagions in $|\overline{\dot{y}}|$ and {\it htp} of participants. In every iteration, we compared the velocity of the participant to the velocity of the co-worker, and classified it as a congruent iteration if the co-worker moved in the same direction as the participant for more than $50\%$ of the iteration time, or otherwise as an incongruent iteration. We then performed the same regression analysis as described above to obtain two slopes for each participant, taking either their congruent, or incongruent iterations. Later, we averaged the difference of the two slopes across the participants to analyze whether congruency affected the \textit{on-line} contagions. The plots of the difference of $|\overline{\dot{y}}|$ and {\it htp} between the congruent and incongruent iterations are shown in Fig.~\ref{fig:cong}.


%%%%%%%%%%%%%%%%%%%%%%%%%%%%%%%%%%%%%%%%%%%%%%%%%%%%%%%%%%%%%%%%%%%%%%%%%%%%%%%%
%                                                                              %
%		                             R E S U L T S                             %
%                                                                              %
%%%%%%%%%%%%%%%%%%%%%%%%%%%%%%%%%%%%%%%%%%%%%%%%%%%%%%%%%%%%%%%%%%%%%%%%%%%%%%%%

%\clearpage

\section{Results}

\subsection{\textit{Off-line} contagions affect mean velocities but not htps}

Our findings agree with the results of recent studies in the past~\cite{Noy:B&C:2009, Kilner:SocialNeuro:2007}, which have shown that \textit{off-line} motor contagions affect the hand movement velocity of participants. We observed (Fig.~\ref{fig:offline}A) a significant positive slope between the mean absolute $y$-velocity ($|\overline{\dot{y}}|$) of participants and the human co-worker in the $H$ condition (median = 0.040, $p = 0.017$, orange plot in Fig.~\ref{fig:offline}A). In the $R_{\text{biol}}$ condition, where the robot co-worker HRP-2Kai made the biological movements, the slope leaned to significance for the $|\overline{\dot{y}}|$ velocity (median = 0.017, Z(38) = 1.86, $p = 0.063$, blue plot in Fig.~\ref{fig:offline}A). Finally in the $R_{\text{nonbiol}}$ condition, when the robot movement was not biological, the results again agreed with previous works and the slope between the $|\overline{\dot{y}}|$ of participants relative to the $|\overline{\dot{y}}|$ of the robot was zero ($p = 0.47$, magenta plot in Fig.~\ref{fig:offline}A). 

Also a positive slope has been observed between the maximum absolute $y$-velocities ($\max |\dot{y}|$) of the participants and of the human co-worker in the $H$ condition (median = 0.54, $p = 0.017$), however this was not present in the robot co-worker conditions ($R_{\text{biol}}$: $p = 0.18$; $R_{\text{nonbiol}}$: $p = 0.29$). Overall, these observations agree and support previous results which showed that the mean velocity of human participants are affected by \textit{off-line} contagions after seeing a human or robot co-worker, although only when the robot co-worker performs biological movements.

Interestingly, the participant's {\it htp}s due to \textit{off-line} contagions remain unaffected, no significant effect was observed. The {\it htp} slopes were observed to be insignificant with human co-worker in the $H$ condition (median = 0.006, $p = 0.06$), as well as the robot co-workers $R_{\text{biol}}$: (median = 0.007, Z(38) = 1.89, $p = 0.06$); $R_{\text{nonbiol}}$: (median = -0.002, Z(17) = -0.18, $p = 0.25$). The $p$ values were marginally insignificant, as illustrate in Fig.~\ref{fig:offline}B.


\begin{figure}[t]
	\centering{\includegraphics[width=0.8\columnwidth]{plots/c2-plots/online_result_vel_htp}}	
	\caption{The \textit{on-line} contagions: Observed changes in the participant's \boldmath{$|\overline{\dot{y}}|$} and \textit{htp} in the $\textit{H}$ (orange plots), $\textit{R}_{\text{biol}}$ (blue plots), $\textit{R}_{\text{nonbiol}}$ (magenta plots) conditions. All p values are Bonferroni corrected.}
	\label{fig:online}
\end{figure}

Note that in our task, it is completely normal to observe a strong positive slope in the $|\overline{\dot{y}}|$, but not in the corresponding {\it htp}s. This is because in our task the participant movements were mainly in the \texttt{YZ} plane, and hence the {\it htp}, which is measured when the participants touches on the touchscreen, and it depends not only on the $y$-velocity, but also the $z$-velocities of the participant. While on the contrary, due to the same reason, any effect induced in the $|\overline{\dot{y}}|$ would partly show up in the {\it htp}s, and this was probably the reason behind the marginal insignificance observed in the participant {\it htp}s.

Finally, we found no effect ($p > 0.1$) on any of the remaining analyzed kinematic variables (maximum movement length ($y_{\max}$), maximum movement height ($z_{\max}$), maximum absolute velocities ($\max|\dot{y}|$, $\max|\dot{z}|$), mean absolute velocity ($|\overline{\dot{z}}|$), maximum accelerations ($\max(\ddot{y})$, $\max(\ddot{z})$), and minimum accelerations ($\min(\ddot{y})$, $\min(\ddot{z})$)) due to the \textit{off-line} contagions, in all three conditions $H$, $R_{\text{biol}}$ and $R_{\text{nonbiol}}$.

\subsection{\textit{On-line} contagions affect htps and not mean velocities}

Our findings strongly suggest that the \textit{on-line} motor contagions are distinct from \textit{off-line} motor contagions. Firstly, we measured a significant effect on the {\it htp} of participants when they worked in parallel with the co-worker, unlike \textit{off-line} contagions. The slope of {\it htp}s was strongly significant both when the participants worked with the robot co-worker who made biological movements (median = 0.017, Z(38) = 3.70, $p = 0.0002$, blue plot in Fig.~\ref{fig:online}B) in the $R_{\text{biol}}$ condition, as well as when they worked with a human co-worker (median = 0.014, $p = 0.0017$, orange plot in Fig.~\ref{fig:online}B) in $H$ condition. As expected, when the robot co-worker's movement were non-biological in nature, no effects were observed in $R_{\text{nonbiol}}$ ($p = 0.777$, magenta plot in Fig.~\ref{fig:online}B).
 
However, in the $H$ condition, we found an effect on the mean absolute $y$-velocity ($|\overline{\dot{y}}|$) of the human participants (median = 0.034, $p = 0.022$, orange plot in Fig.~\ref{fig:online}A), but we didn't find this effect in the $R_{\text{biol}}$ condition (median = 0.013, Z(38) = 0.13, $p = 0.90$, blue plot in Fig.~\ref{fig:online}A), nor in the $R_{\text{nonbiol}}$ condition ($p = 0.39$, magenta plot in Fig.~\ref{fig:online}A). Suggesting that this effect was overall absent with the robot co-worker. At last, again with human co-worker, we noticed some effects in the $\max(\ddot{y})$ and $\max(\ddot{z})$, but it was completely absent in both of the robot co-worker conditions.


\begin{figure}[t]
	\centering{\includegraphics[width=.7\columnwidth]{plots/c2-plots/online_result_cong}}
	\caption{Effect of congruency on \textit{on-line} contagions: the difference in slopes, between the velocity congruent and incongruent iterations across participants, was zero for both the \boldmath{$|\overline{\dot{y}}|$} and \textit{htp}s of participants during the observation of the human ($\textit{H}$, orange plot) condition and robot co-worker ($\textit{R}_{\text{biol}}$, blue plot) condition. The lack of effect difference suggests that the \textit{on-line} contagion does not affect the movement velocities in our study.}
	\label{fig:cong}
\end{figure}

Our findings in the $R_{\text{biol}}$ condition favors the notion that \textit{off-line} contagions affect a participant's {\it htp} but not a participant's $|\overline{\dot{y}}|$. On the other hand, in the $H$ condition, the effects on both {\it htp} and $|\overline{\dot{y}}|$ were observed. Therefore to settle this conflict, we consequently measured if the results in the $H$ condition were coupled; \texttt{i.e.}, whether the $|\overline{\dot{y}}|$ was indeed affected in the $H$ condition, or whether it was a consequence of the effect on the {\it htp}. In the $H$ conditions, we separated and collected the movement iterations depending on whether the participant's movement was predominantly congruent ({\it cong} iterations), or incongruent ({\it in-cong} iterations) with an observed (co-worker's) movement (see Subsection~\nameref{congSec} for details), and compared the {\it on-line} contagions in these two types of iterations separately. By {\it cong} iterations we mean, when the co-worker's movement direction corresponds to that of the participant, and the {\it in-cong} iterations, when the movements direction do not correspond. In accordance with the previous studies~\cite{ Kilner:SocialNeuro:2007, Bisio:PlosOne:2010, Noy:B&C:2009}, we hypothesized that if the \textit{on-line} contagions affect the $|\overline{\dot{y}}|$, then the contagion strength (\texttt{i.e.} the signed slope), would be significantly different between the {\it cong} iterations and the {\it in-cong} iterations. Otherwise, if the {\it on-line} contagions are in the {\it htp}, which is a time unit, the congruency of the observed movement (relative to the participant's own movement) should not change the strength of the contagions.

Our data analysis and results stipulate no difference in the $|\overline{\dot{y}}|$ and {\it htp} slopes in the {\it cong} and {\it in-cong} iterations of the $H$ condition ($p = 0.26$ and $p = 0.73$, orange plots in Fig.~\ref{fig:cong} respectively). Likewiese, no such difference was either observed between the slopes of $|\overline{\dot{y}}|$  and {\it htp} in the {\it cong} and {\it in-cong} iterations of the $R_{\text{biol}}$ condition ($p = 0.76$ and $p = 0.59$, blue plots in Fig.~\ref{fig:cong} respectively). Therefore, these results strongly emphasize that the \textit{on-line} contagions predominantly affect the participant's {\it htp}s but not velocity.


%%%%%%%%%%%%%%%%%%%%%%%%%%%%%%%%%%%%%%%%%%%%%%%%%%%%%%%%%%%%%%%%%%%%%%%%%%%%%%%%
%                                                                              %
%             D I S C U S S I O N and C O N C L U S I O N                      %
%                                                                              %
%%%%%%%%%%%%%%%%%%%%%%%%%%%%%%%%%%%%%%%%%%%%%%%%%%%%%%%%%%%%%%%%%%%%%%%%%%%%%%%%

%\clearpage

\section{Discussion}

We initially asked three particular questions in the beginning of this Chapter, regarding the affects of a human or humanoid robot co-worker's action observation on one's behavior, during and after the same action observations. Our findings agree with the previous works on \textit{off-line} motor contagion and show similar results that the mean absolute velocity of human participants in the predominant movement direction (\texttt{Y} direction, in this study) are implicitly affected after observing a co-worker (both human and robot), but with the robot co-worker, this effect was present only when the robot made biological movements (Fig.~\ref{fig:trial}B, blue plot). On the contrary, due to \textit{off-line} contagions, we found minimal effects on the participant's {\it htp}s (Fig.~\ref{fig:offline}B). While due to the \textit{on-line} contagions, we primarily observed the effects in the participant {\it htp}s during both, when working with a human or robot co-worker, however with robot co-worker, again only when the robot made biological movements (Fig.~\ref{fig:online}B). We also observed an affect on the mean absolute $y$-velocity of participants when they worked with human co-workers (Fig.~\ref{fig:online}A), but on the contrary, the \textit{iteration}s congruency analysis (section~\nameref{congSec}) strongly emphasizes that this effect was in fact a residual effect on the {\it htp}. To summarize, our results suggest that both \textit{on-line} and \textit{off-line} contagions affect distinct movement features of the human participant from the observation of same movement. The \textit{on-line} contagions are more frequent in the frequency or rhythm (quantified by {\it htp}) of the movements, while the \textit{off-line} contagions mostly affect velocity.

In this study we quantified the \textit{on-line} contagions as the relation (slope) between the \textit{difference} of the human participant's movement feature when working with the co-worker compared to working alone, and the co-workers feature. This difference extracts the \textit{off-line} effects in the participant behavior. Which in fact arises due to the observation of a previous and different co-worker movements. Therefore it is important to note that the insufficiency to obtain a particular effect in the \textit{on-line} contagions analysis, doesn't necessarily mean that the effect is absent during these observations of the co-worker. Rather here the analysis of \textit{on-line} contagions represents particularly the effects that changed when working alone, in compared to when working parallel to a co-worker.

Interestingly, our findings suggests that the nature of the co-worker, (human or a robot), \textit{tend} to influence the \textit{off-line} contagions significantly more than the \textit{on-line} contagions. While with the human co-worker, strong \textit{off-line} contagions were measured in the participant's $|\overline{\dot{y}}|$ ($p = 0.017$, Fig.~\ref{fig:offline}A), but with the robot co-worker, this effect seemed diminished ($p = 0.063$, Fig.~\ref{fig:offline}A). Although the difference of these effects between the two conditions ($p = 0.34$) weren't significant enough to conclude definitely. However, the effect on the participant's {\it htp}s was clearly visible due to \textit{on-line} contagions, both with the human co-worker ($p = 0.0017$, Fig.~\ref{fig:online}B) and robot co-worker ($p = 0.0002$, Fig.~\ref{fig:online}B), and as well these affects were not different from each other ($p = 0.62$). Therefore as our results suggest, compared to the \textit{on-line} contagions, perhaps the \textit{off-line} contagion is more sensitive to the nature of the co-worker. There is also a factor of age, physical and behavioral charateristics of the partner, which may indirectly affect these two motor contagions, however this study didn't deal with these factors sufficiently and perhaps should be a topic of discussion for future research. Also, both the \textit{off-line} and \textit{on-line} contagions were observed to be sensitive to the behavior of the co-worker (human and robot), while with the robot co-worker only when the movement's of the robot was biological in nature. 

Finally, the overall observations made in this Chapter emphasize on our hypothesis that distinct motor contagions are induced in human participant's \emph{during} the observation of a co-worker (\textit{on-line} contagions) and as well as \emph{after} the observations of a co-worker (\textit{off-line} contagions). These observed distinctions in the affected movement features and the sensitivity of these effects to the nature of the co-worker provide a better understanding on how human movements may be influenced by the robot co-workers working near them. This insights could be crucial to the physical and behavioral design of robots working near humans.


In the next Chapter, we further explored the topic of motor contagions between human and humanoid robot co-workers and our findings suggest that by exploiting motor contagions, one can influence the performance of human co-worker and while where ethically valid, these motor contagions may also be used to improve worker performance speed and hence productivity in a industrial task.

\clearpage % end of Roman