% ======================================================================= %
% ======================================================================= %

{\color{blue}\chapter*{Conclusion - incomplete}}
\addcontentsline{toc}{chapter}{Conclusion}
\pagestyle{plain}

%Overview
To conclude this thesis, our achieved results contributed in the broad field of human robot interactions (specially with the humanoid robot) both at a safer distance and in close proximity, namely during a human-robot interaction (HRI) and physical human robot interaction (pHRI) respectively. The work done in this thesis was about the interactions between human and biped humanoid robot as `co-workers' in the industrial scenarios. We started with the non-physical human-robot interaction scenario based on a industrially inspired \textit{Pick-n-Place} task example and then advanced towards the physical human-robot interactions with an example of human humanoid robot dual arm bi-directional object handover.

Since beginning, this thesis was divided into two parts. In the context of \textit{non-physical} human-robot interactions, the studies conducted in 1$^{st}$ part of this thesis (Chapter~\ref{distinct motor contagion} and Chapter~\ref{more than just co-workers}) were related to the behavioral effects of motor contagions and motivated by the `implicit' social interactions between human and humanoid robot co-workers. While in the context of \textit{physical} human-robot interactions, the 2$^{nd}$ part of this thesis (Chapter~\ref{handover chapter}) was motivated by the physical manipulations of object between human and humanoid robot co-workers in close proximity using humanoid robot whole-body control framework.

We examined an empirical repetitive industrial task in which a human participant and a humanoid robot co-worker near each other. We primarily chose cyclic and repetitive \textit{pick-n-place} task for the experiments in Chapters \ref{distinct motor contagion} and~\ref{more than just co-workers}, as we wanted a task that is simple but rich and could represent several industrial co-worker scenarios. We found that this is one of the most common task in current industrial platforms where robots are often employed.

%Ro-man
In Chapter~\ref{distinct motor contagion}, our results and findings suggest that \textit{on-line} contagions affect participant's movement frequency while the \textit{off-line} contagions affect their movement velocity. Also \textit{off-line} motor contagions were mainly notable after observing human co-worker but the effects of \textit{on-line} contagions were equal with both human and humanoid robot co-workers. Therefore, perhaps the \textit{off-line} contagion is more sensitive to the nature of the co-worker. These two contagions were also observed to be sensitive to the behavioral features of both co-workers, but with robot co-worker, these motor contagions were induced only when robot movement's were biological in nature. Note that this study didn't consider effects of factors such age, physical or behavioral characteristics of the partner co-worker. They may have had indirectly affected these two motor contagions, perhaps its an interesting topic of discussion and need to be explored in future research. Finally, the overall observations made in this Chapter emphasize on our hypothesis that distinct motor contagions are induced in human participant's \emph{during} the observation of a co-worker (\textit{on-line} contagions) and as well as \emph{after} the observations of same co-worker (\textit{off-line} contagions).

%Plos one
%We systematically varied the behavior, specifically frequency of robot movements and examined whether and how the frequency of movements by the human participants, and their task accuracy, is affected by the presence of the robot. 

In Chapter~\ref{more than just co-workers}, we further explored our findings from Chapter~\ref{distinct motor contagion} and under the same experimental task and setup along with the addition of few more conditions. Primarily, our findings suggest that the presence of a humanoid robot co-worker (or a human co-worker) can influence the performance frequencies of human participants. We observed that participants become slower with a slower co-worker, but also faster with faster co-workers. We also argued that the performance has to be measured considering together, both the task speed (or frequency) as well as task accuracy. Some previous studies in motor control have shown that motor noise often constrain human movements, which eventually increases with the enormity of motor commands in the muscles~\cite{Harris:Nature:1998}. This consequently leads to the trade-off between the speed and accuracy of normal and usual daily life movements~\cite{Fitts:JEP:1954}. However, by regulating the arm impedance by muscle co-contraction, one can also modulate the movement accuracy~\cite{Burdet:nature:2001, Franklin:JoN:2008, Ganesh:RAS:2013}. Therefore here we showed how the touch accuracy of the participants have changed alongside the contagions in their {\it htp} and hence the performance of the human co-worker during the task. We also investigated the effects of physical form, by adding two conditions where both human and robot co-worker's head and torso were covered, and human participants were only able to see visible moving arm of the co-worker. Our findings suggest that the presence of a humanoid co-worker can affect human performance, but only when it's humanoid form is visible. Moreover, this effect was supposedly increased with the human participants having prior robot experience. Finally, our results show that human task frequency, but not task accuracy, is affected by the observation of a humanoid robot co-worker, provided the robot's head and torso are visible. Note that in Chapter~\ref{more than just co-workers}, we quantify the participants level of the robot exposure by a self perceived robot exposure score by themselves rather than the actual robot exposure. As an actual standard robot exposure questionnaire to measure this effect is currently absent and the development of one can be useful to understand how this effects, such as the one we highlight here, vary over time.

Overall the findings mentioned and discussed in Chapters~\ref{distinct motor contagion} and~\ref{more than just co-workers} highlights several new features of motor contagions, but also open new questions for future research. One can find these results useful for customizing the design of robot co-workers in industries and sports in future studies by moderating or exploiting these contagions. While if ethically allowed, these contagions could be utilized to improve co-workers performance speed and hence productivity.

%Handover
In Chapter~\ref{handover chapter},

%The main way to improve our proactive control scheme would be to improve the walking capabilities of the robot, both in velocity and stability. C

\clearpage % end of Conclusion
\pagestyle{fancy}












