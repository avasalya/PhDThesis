% ======================================================================= %
% ======================================================================= %

{\color{blue}\chapter*{Conclusion \textit{incomplete**}}}
\addcontentsline{toc}{chapter}{Conclusion \textit{incomplete**}}

\textbf{THE 1st part/chapter of this thesis}


\textit{In this study we wanted to choose a task that is simple, yet rich, and is representative of many industrial co-worker scenarios. We found that (repetitive) pick and place tasks to be the most common industrial tasks in which robots are employed. We therefore chose to start with a cyclic touch task in this experiment.}

roman::\textit{ Our findings suggest that on-line contagions affect participant's movement frequency while the \textit{off-line} contagions mainly affect their movement velocity. Moreover, on-line contagions were equally effective with either a human or a humanoid robot co-worker, and the \textit{off-line} contagions were notable after observing another human. Finally these results suggest that the actions by a humanoid robot co-worker can induce distinct effects on human behaviors, during and after observation.}


plosone:: \textit{Our results show that human task frequency, but not task accuracy, is affected by the observation of a humanoid robot co-worker, provided the robot's head and torso are visible.
}




\textit{Studies in motor control have exhibited that human movements are constrained by motor noise, which increases with the magnitude of motor commands in the muscles~\cite{Harris:Nature:1998}. In the case of `regular' and automatic movements in daily life, this leads to a trade-off between the speed and accuracy of the movement~\cite{Fitts:JEP:1954}. However, the accuracy of movements is also modulated by the regulation of arm impedance by muscle co-contraction~\cite{Burdet:nature:2001, Franklin:JoN:2008, Ganesh:RAS:2013}. As mentioned earlier, to comment on the task performance of the human co-worker, we next analyzed whether and how the touch accuracy of the participants changed alongside the contagions in their {\it htp}. 
}

to address these issues, we further explored our findings from Chapter~\ref{distinct motor contagion} and  ------
\textit{we examined an empirical repetitive industrial task in which a human participant and a humanoid robot work near each other. We systematically varied the behavior, specifically frequency of robot movements and examined whether and how the frequency of movements by the human participants, and their task accuracy, is affected by the presence of the robot. To investigate the effect of physical form, we added conditions where the robot co-worker torso and head were covered, and only the moving arm was visible to the human participants. Finally, in order to compare the humanoid co-worker to a human co-worker, we also checked how the effects on the participants changed with a human co-worker, with and without his/her torso and head visible. To anticipate our results, we found that the presence of a humanoid co-worker can affect human performance, but only when it's humanoid form is visible. Furthermore, the effect was observed to increase with prior robot experience by the humans.}



\textbf{the 2nd part/final chapter of this thesis}

\clearpage % end of Conclusion
