% ======================================================================= %
% ======================================================================= %

{\color{blue}\chapter*{Conclusion}}
\addcontentsline{toc}{chapter}{Conclusion}
\pagestyle{plain}

%Overview
To conclude this thesis, our achieved results contributed in the broad field of human-robot interactions (especially with the humanoid robot) both at a safer distance and in close proximity, namely during a human-robot interaction (HRI) and physical human-robot interaction (pHRI) respectively. The work done in this thesis was about the interactions between human and biped humanoid robot as `co-workers' in the industrial scenarios. We started with the non-physical human-robot interaction scenario based on an industrially inspired \textit{Pick-n-Place} task example and then advanced towards the physical human-robot interactions with an example of human, humanoid bi-manual bi-directional object handover.

Since the beginning, this thesis was divided into two parts. In the context of \textit{non-physical} human-robot interactions, the studies conducted in 1$^{st}$ part of this thesis (Chapter~\ref{distinct motor contagion} and Chapter~\ref{more than just co-workers}) were related to the behavioural effects of motor contagions and motivated by the `implicit' social interactions between human and humanoid co-workers. While in the context of \textit{physical} human-robot interactions, the 2$^{nd}$ part of this thesis (Chapter~\ref{handover chapter}) was motivated by the physical manipulations of object and handover between nearby human and humanoid using robot whole-body control framework and locomotion.


We examined an empirical repetitive industrial task in which a human participant and a humanoid co-worker near each other. We primarily chose cyclic and repetitive \textit{pick-n-place} task for the experiments in Chapters \ref{distinct motor contagion} and~\ref{more than just co-workers}, as we wanted a task that is simple but rich and could represent several industrial co-worker scenarios. We found that this is one of the most common tasks in current industrial platforms where robots are often employed.


%Ro-man
In Chapter~\ref{distinct motor contagion}, our results and findings suggest that \textit{on-line} contagions affect participant's movement frequency while the \textit{off-line} contagions affect their movement velocity. Also, \textit{off-line} motor contagions were mainly notable after observing human co-worker, but the effects of \textit{on-line} contagions were equal with both human and humanoid co-workers. Therefore, perhaps the \textit{off-line} contagion is more sensitive to the nature of the co-worker. These two contagions were also observed to be sensitive to the behavioural features of both co-workers, but with robot co-worker, these motor contagions were induced only when robot movements were biological. Note that this study did not consider the effects of factors such as age, physical or behavioural characteristics of the partner co-worker. They may have had indirectly affected these two motor contagions, perhaps its an interesting topic of discussion and need to be explored in future research. Finally, the overall observations made in this Chapter emphasize on our hypothesis that distinct motor contagions are induced in human participant's \emph{during} the observation of a co-worker (\textit{on-line} contagions) and as well as \emph{after} the observations of same co-worker (\textit{off-line} contagions).



%Plos one
In Chapter~\ref{more than just co-workers}, we further explored our findings from Chapter~\ref{distinct motor contagion} and under the same experimental task and set up along with the addition of a few more conditions. Primarily, our findings suggest that the presence of a humanoid co-worker (or a human co-worker) can influence the performance frequencies of human participants. We observed that participants become slower with a slower co-worker, but also faster with a faster co-worker. We also argued the performance has to be measured considering together, both task speed (or frequency) as well as task accuracy. Some previous studies in motor control have shown that motor noise often constrains human movements, which eventually increases with the enormity of motor commands in the muscles~\cite{Harris:Nature:1998}. This consequently leads to the trade-off between the speed and accuracy of ordinary and usual daily life movements~\cite{Fitts:JEP:1954}. However, by regulating the arm impedance by muscle co-contraction, one can also modulate the movement accuracy~\cite{Burdet:nature:2001, Franklin:JoN:2008, Ganesh:RAS:2013}. Therefore here we showed how touch accuracy of participants have changed alongside the contagions in their {\it htp} and hence performance of the human co-worker during the task. We also investigated the effects of physical form, by adding two conditions where both human and robot co-worker's head and torso were covered, and human participants were only able to see visible moving arm of the co-worker. Our findings suggest that the presence of a humanoid co-worker can affect human performance, but only when its humanoid form is visible.

Moreover, this effect was supposedly increased with the human participants having prior robot experience. Finally, our results show that human task frequency, but not task accuracy, is affected by the observation of a humanoid co-worker, provided the robot's head and torso are visible. Note that in Chapter~\ref{more than just co-workers}, we quantify the participants level of the robot exposure by a self-perceived robot exposure score by themselves rather than the actual robot exposure. As an actual standard robot exposure questionnaire to measure this effect is currently absent, and the development of one can be useful to understand how this effects, such as the one we highlight here, vary over time.

Overall the findings mentioned and discussed in Chapters~\ref{distinct motor contagion} and~\ref{more than just co-workers} highlights several new features of motor contagions, but also open new questions for future research. One can find these results useful for customizing the design of robot co-workers in industries and sports in future studies by moderating or exploiting these contagions. While if ethically allowed, these contagions could be utilized to improve co-workers performance speed and hence productivity.




%Handover
In Chapter~\ref{handover chapter}, we designed a framework to handover an object between human and robot co-workers in the context of pHRI. We concentrated our efforts towards developing a simple yet robust and efficient handover framework. We introduced an intuitive bi-directional object handover \textit{routine} between a human-humanoid dyad using whole-body control (WBC) and locomotion. Throughout this chapter, the problem of bi-directional object handover between human and humanoid co-worker was treated as one-shot continuous fluid motion. Initially, we started by designing a general framework and within it developed models to predict human hand position to converge at the handover location (\nameref{prediction_model}) along with estimating the grasp configuration of an object and active human hand during handover \textit{cycle}s (\nameref{orientation_model}). We also designed a model (\nameref{interaction model}) to minimize the interaction forces during the handover of an unknown mass object along with minimizing the overall duration of object handover \textit{routine}. We designed these models to answer three important questions related to human-robot object handover ---\textbf{when} (\textit{timing}), \textbf{where} (\textit{position in space}) and \textbf{how} (\textit{orientation and interaction forces}) during a handover \textit{routine}. 

Within this handover framework, using HRP-2Kai, we first tested and confirmed the feasibility of these models under the scenario where human and robot co-workers always use their right hand and left end-effector, respectively. Later a generalized framework was presented (\nameref{both hands individual}) and tested where both human and robot co-workers were able to choose either of their hand to handover or exchange the object. In addition, thanks to the native low-level (\nameref{QPController}) of our robot HRP-2Kai and by utilizing whole-body control configuration, we were able to extend further our handover framework, which allowed the robot to use both hands simultaneously during the object handover \textit{routine} (\nameref{both hands together}). 

Furthermore, as mentioned earlier, none of the previous work on the human-robot dyad considered object handover and `walking' in a single framework using a \textit{biped} humanoid platform such as HRP-2Kai. Therefore for a proactive handover of an object between human and robot we believed it was essential to consider the possibility of a robot taking a step to handover or exchange an object with the human co-worker, in scenarios where short-distance travel is required. Hence, we explored the full capabilities of HRP-2Kai and added a scenario where the robot needs to proactively take a few steps in order to handover or exchange the object between its human co-worker. This scenario had been implemented on HRP-2Kai and tested during both when human-robot dyad uses either single or both hands simultaneously. 


Finally, to conclude our preliminary testing of the complete handover framework under above-mentioned scenarios, including locomotion. We confirm that our bi-directional object handover framework is intuitive and adaptable to several objects of distinguishable physical properties (shape, size and mass), including industrial tools. It only needs the information of handover object's shape and size, though knowing the mass of the object is not important in the beginning. We confirmed the feasibility of our handover framework under several objects with mass ranges from [$0.17$ to $1.1$] kg, during above-mentioned scenarios. However, over the course of several handover \textit{routine} trials, we report that the calculated object mass is not accurate and has marginal error of $ \pm10 $\% compared to object's actual mass and needs further improvement on better estimation of the inertial forces at play or  better ways to remove offsets from force sensors although it didn't affect the optimal threshold which is vital at the time of object handover from robot to human co-worker.

Though overall our method allows us not to stop the motion of end-effector and still being able to handover (both ways) the object (see \nameref{FSM}), however, if the handover occurs while both human and robot end-effectors are moving then this problem of handover would be extended to the problem of object collaboration and manipulation, which is already being studied extensively in our research group \cite{bussy2012proactive, agravante2016walking, agravante2013human, bussy2012human, agravante2014collaborative, Agravante2019}. Therefore we concentrated solely on the proactive bi-directional handover problem and hence decided to reduce the end-effectors velocity (${}^\text{ef}v\simeq0$) at the time of handover. 

By adding a dimension of step-walk locomotion, We did not focus on the problem of motion planning or navigation in a large cluttered environment~\cite{mainprice2012sharing, vahrenkamp2009humanoid, kim2004advanced}. However, instead, we concentrated our efforts to solve and optimize object handover problem which requires immediate shared efforts between human-robot dyad in a small space where few steps are necessary and enough for a comfortable and convenient object handover. Our proposed method is simple but effective to take advantage of the humanoid robot and deal with the problem of bi-directional object handover using robot whole-body control. However, note that in the cases when human co-worker decides to step backwards, then for sometime after the forward step-walk is triggered, the interpersonal distance $ D $ would become larger than 1.2 meters (since human is away from the robot), therefore in order for this model to work, the human co-worker shouldn't take more than few steps backward.

Though at the moment, step-walk is triggered only when interpersonal distance between the co-workers becomes greater than $ 125\% <= D <= 150\% $, however more possibilities including the estimation of direction in which human co-worker may wish to move (laterally) and handover the object will be considered in the future scope of this study. At last, in the near future, a human-user study will be conducted shortly after further improving the performance of the overall proactive handover framework with step-walk locomotion.


\clearpage % end of Conclusion
\pagestyle{fancy}









