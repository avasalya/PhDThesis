{\color{blue}\chapter*{\centering{Résumé de la thèse}}}
\addcontentsline{toc}{chapter}{Résumé de la thèse}
\thispagestyle{empty}

%Human-robot interaction is an emerging field which deals with the study and research of interactions between humans and robots~\cite{goodrich2008human}. The work done in this thesis is about the interactions between human and humanoid robot HRP-2Kai as co-workers in the industrial scenarios. In the context of \textit{non-physical} human-robot interactions (HRI), the studies conducted in the 1$^\text{st}$ part of this thesis investigated the implicit behavioral effects of humanoid robot on the behavior of human co-workers during an industrially inspired \textit{Pick-n-Place} task paradigm. In the context of \textit{physical} human-robot interactions (pHRI), we developed a novel bi-manual object handover framework using robot whole-body control and locomotion in the 2$^\text{nd}$ part of this thesis.

L'interaction humain-robot est un domaine émergent qui traite de l'étude et de la recherche des interactions entre les humains et les robots~\cite{goodrich2008human}. Le travail effectué dans cette thèse porte sur les interactions entre l'homme et le robot humanoïde HRP-2Kai en tant que collaborateurs dans les scénarios industriels. Dans le contexte des interactions human-robot (HRI) non physiques, les études menées dans la première partie de cette thèse ont examiné les effets comportementaux implicites du robot humanoïde sur le comportement des collègues humains au cours d'un paradigme de tâches d'inspiration industrielle \textit{Pick-n-Place}. Dans le contexte des interactions homme-robot (pHRI), nous avons développé un nouveau framework de transfert d'objets bi-manuel utilisant le contrôle du corps entier et la locomotion d'un robot dans la 2ème partie de cette thèse.\\

%%% roman
%\paragraph*{\LARGE {Distinct motor contagions \\}\\}

%When an individual (human and robot) performs an action followed by the observation of someone's action, behavioral implicit effects such as motor contagions causes certain features (kinematics parameters, goal or outcome) of that action to become similar to the observed action. However previous studies on motor contagions have examined  effects induced either during the observation of action or after but never together therefore it remains unclear whether and how these effects are distinct from each other.

\paragraph*{\LARGE {Distinct motor contagions \\}\\}

Lorsqu'un individu (humain et robot) effectue une action suivie de l'observation de l'action d'une personne, des effets comportementaux implicites tels que des contagions motrices font que certaines caractéristiques (paramètres cinématiques, but ou résultat) de cette action deviennent semblables à l'action observée. Cependant, des études antérieures sur les contagions motrices ont examiné les effets induits soit pendant l'observation de l'action, soit après, mais jamais ensemble, c'est pourquoi il n'est donc pas clair si ces effets sont distincts les uns des autres et en quoi ils sont different.


%In Chapter~\ref{distinct motor contagion}, during an industrially inspired repetitive \textit{Pick-n-Place} movement task paradigm, we examined the effect of motor contagions induced in participants during (we call it \textit{on-line} contagions) and after (\textit{off-line} contagions) the observation of the same movements performed by a human, or a humanoid robot co-worker. We specifically examined three important questions:

Dans le chapitre~\ref{distinct motor contagion}, au cours d'un paradigme de tâche de mouvement répétitif \textit{Pick-n-Place} d'inspiration industrielle, nous avons examiné l'effet des contagions motrices induites chez les participants pendant (nous appelons cela \textit{on-line} contagions) et après (\textit{off-line} contagions) les mêmes mouvements effectués par un collègue humain ou un robot humanoïde. Nous avons examiné en particulier trois questions importantes :



%\begin{enumerate}
%	\item Can on-line and off-line contagions from the observation of a same movement affect different movement features of the human participant?
%	\item How do the strengths of the on-line and off-line contagions vary with the nature of the co-worker (ie. if human or robot), and the behavior of the co-worker?
%	\item Consequently, are the on-line and off-line contagions different, or do they constitute the same effect observed at different instances?
%\end{enumerate}

\begin{enumerate}
	\item Les contagions en ligne et hors ligne résultant de l'observation d'un même mouvement peuvent-elles affecter différentes caractéristiques de mouvement du participant humain ?
	\item Comment les forces des contagions en ligne et hors ligne varient-elles selon la nature du collègue (c.-à-d. s'il s'agit d'un humain ou d'un robot) et le comportement du collègue de travail ?
	\item Par conséquent, les contagions en ligne et hors ligne sont-elles différentes ou constituent-elles le même effet observé à différents moments ?
\end{enumerate}



%Our results and findings suggest that \textit{on-line} contagions affect participant's movement frequency while the \textit{off-line} contagions affect their movement velocity. Also \textit{off-line} motor contagions were mainly notable after observing human co-worker but the effects of \textit{on-line} contagions were equal with both human and humanoid robot co-workers. Therefore, perhaps the \textit{off-line} contagion is more sensitive to the nature of the co-worker. These two contagions were also observed to be sensitive to the behavioral features of both co-workers, but with robot co-worker, these motor contagions were induced only when robot movement's were biological in nature.  Finally, the overall observations made in this Chapter emphasize on our hypothesis that distinct motor contagions are induced in human participant's \emph{during} the observation of a co-worker (\textit{on-line} contagions) and as well as \emph{after} the observations of same co-worker (\textit{off-line} contagions).

Nos résultats et conclusions suggèrent que les contagions \textit{on-line} affectent la fréquence de mouvement du participant tandis que les contagions \textit{off-line} affectent leur vitesse de mouvement. De plus, les contagions motrices \textit{off-line} étaient principalement notables après l'observation d'un collègue humain, mais les effets des contagions \textit{on-line} étaient les mêmes que ceux des collègues humains et des robots humanoïdes. Par conséquent, la contagion \textit{off-line} est peut-être plus sensible à la nature du collègue. Ces deux contagions ont également été observées comme étant sensibles aux caractéristiques comportementales des deux collègues, mais avec les robots, ces contagions motrices n'ont été induites que lorsque les mouvements du robot étaient de nature biologique.  Enfin, les observations générales faites dans ce chapitre mettent l'accent sur notre hypothèse que des contagions motrices distinctes sont induites chez les participants humains lors de l'observation d'un collègue de travail (\textit{on-line} contagions) et aussi bien que les observations du même collègue de travail (\textit{off-line} contagions).


%%% plosone
%\paragraph*{\LARGE {Motor contagion influences human co-worker performance \\}\\}

%Does the presence of a humanoid robot co-worker influence the performance of humans around it? While the past studies have examined how the effects of induced motor contagions due to the observation of human and robot movements have affected either human co-worker's movement velocity or how it affected movement variance but never both together. Therefore we argue that since precision in movements along with speed is the key in most industrial tasks, hence it is necessary to consider both task accuracy and task speed to accurately measure the performance in a task.

\paragraph*{\LARGE {Motor contagion influences human co-worker performance  \\}\\}

La présence d'un collègue robot humanoïde influence-t-elle la performance des humains qui l'entourent ? Alors que les études passées ont examiné comment les effets des contagions motrices induites par l'observation des mouvements de l'homme et du robot ont affecté soit la vitesse de mouvement du collègue humain, soit la variance du mouvement, mais jamais les deux ensemble. Par conséquent, nous soutenons que puisque la précision dans les mouvements avec la vitesse est la clé dans la plupart des tâches industrielles, il est donc nécessaire de considérer à la fois la précision et la vitesse de la tâche pour mesurer précisément la performance dans une tâche.




%In Chapter~\ref{more than just co-workers}, under the same paradigm and repetitive industrial task and along with the addition of few more conditions, we systematically varied the robot behavior, and observed whether and how the performance of a human participant is affected by the presence of a humanoid co-worker. We also investigated the effect of physical form of humanoid robot where torso and head were covered, and only the moving arm was visible to the human participants. Later, we compared these behaviors with a human co-worker, and examined how the observed behavioral affects scale with experience of robots. 

Dans chapitre~\ref{more than just co-workers}, sous le même paradigme et tâche industrielle répétitive et avec l'ajout de quelques conditions supplémentaires, nous avons systématiquement varié le comportement du robot, et observé si et comment la performance d'un participant humain est affectée par la présence d'un collaborateur humanoïde. Nous avons également étudié l'effet de la forme physique du robot humanoïde où le torse et la tête étaient couverts, et où seul le bras mobile était visible pour les participants humains. Plus tard, nous avons comparé ces comportements avec ceux d'un collègue humain et examiné comment le comportement observé affecte l'échelle avec l'expérience des robots. 





%Primarily, our findings suggest that the presence of a humanoid robot co-worker (or a human co-worker) can influence the performance frequencies of human participants. We observed that participants become slower with a slower co-worker, but also faster with faster co-worker. We also argued that the performance has to be measured considering together, both the task speed (or frequency) as well as task accuracy. We showed how the touch accuracy of the participants have changed alongside the contagions in their {\it htp} and hence the performance of the human co-worker during the task.
 
Principalement, nos résultats suggèrent que la présence d'un collègue robot humanoïde (ou d'un collègue humain) peut influencer les fréquences de performance des participants humains. Nous avons observé que les participants deviennent plus lents avec un collègue plus lent, mais aussi plus rapides avec un collègue plus rapide. Nous avons également soutenu que le rendement doit être mesuré en considérant ensemble la vitesse (ou la fréquence) et l'exactitude des tâches. Nous avons montré comment la précision du toucher des participants a changé en même temps que les contagions dans leur {\it htp} et donc la performance du collaborateur humain pendant la tâche.




%We also investigated the effects of physical form, by adding two conditions where both human and robot co-worker's head and torso were covered, and human participants were only able to see visible moving arm of the co-worker. Our findings suggest that the presence of a humanoid co-worker can affect human performance, but only when it's humanoid form is visible. Moreover, this effect was supposedly increased with the human participants having prior robot experience. Finally, our results show that human task frequency, but not task accuracy, is affected by the observation of a humanoid robot co-worker, provided the robot's head and torso are visible and robot made biological movements.

Nous avons également étudié les effets de la forme physique, en ajoutant deux conditions dans lesquelles la tête et le torse de l'homme et du robot étaient couverts, et les participants humains ne pouvaient voir que le bras mobile visible de leur collègue. Nos résultats suggèrent que la présence d'un collègue humanoïde peut affecter la performance humaine, mais seulement lorsque sa forme humanoïde est visible. De plus, cet effet a été supposément accru par le fait que les participants humains avaient déjà fait l'expérience d'un robot. Enfin, nos résultats montrent que la fréquence des tâches humaines, mais non la précision des tâches, est affectée par l'observation d'un collègue robot humanoïde, à condition que la tête et le torse du robot soient visibles et que les mouvements biologiques du robot soient réalisés.

%%% handover
%\paragraph*{\LARGE {Bi-manual and locomotion synchronized bi-directional object handover \\}\\}

%In Chapter~\ref{handover chapter}, we designed a framework to handover an object between human and robot co-workers in the context of pHRI. We concentrated our efforts towards developing a simple yet robust and efficient handover framework. We introduced an intuitive bi-directional object handover \textit{routine} between human-humanoid co-workers using whole-body control (WBC) and locomotion. Throughout this Chapter, the problem of bi-directional object handover between human-humanoid co-workers was treated as one-shot continuous fluid motion. Initially we started by designing a general framework and within it developed models to predict human hand position to converge at the handover location, along with estimating the grasp configuration of object and active human hand during handover \textit{cycle}s. We also designed a model to minimize the interaction forces during the handover of an unknown mass object along with minimizing the overall duration of object handover \textit{routine}. We designed these models to answer three important questions related to human robot object handover ---\textbf{when} (\textit{timing}), \textbf{where} (\textit{position in space}) and \textbf{how} (\textit{orientation and interaction forces}) during a handover \textit{routine}.

\paragraph*{\LARGE {Bi-manual and locomotion synchronized bi-directional object handover \\}\\}

Dans le chapitre~\ref{handover chapter}, nous avons conçu un framework pour le transfert d'un objet entre collègues humains et robots dans le contexte de pHRI. Nous avons concentré nos efforts sur l'élaboration d'un framework de transfert simple mais robuste et efficace. Nous avons introduit un transfert bidirectionnel intuitif d'objets \textit{routine} entre collègues humains et humanoïdes utilisant le contrôle du corps entier (WBC) et la locomotion. Tout au long de ce chapitre, le problème du transfert bidirectionnel d'objets entre collègues humains et humanoïdes a été traité comme un mouvement fluide continu et ponctuel. Au départ, nous avons commencé par concevoir un framework général dans lequel nous avons développé des modèles pour prédire la position de la main humaine convergeant au point de transfert, ainsi que pour estimer la configuration de saisie de l'objet et de la main humaine active pendant le transfert \textit{cycle}s. Nous avons également conçu un modèle pour minimiser les forces d'interaction lors de la remise d'un objet de masse inconnu ainsi que pour minimiser la durée totale de la remise d'un objet \textit{routine}. Nous avons conçu ces modèles pour répondre à trois questions importantes liées à la remise d'objet robot humain ---\textbf{quand} (\textit{timing}), \textbf{là où} (\textit{position in space}) and \textbf{comment} (\textit{orientation et forces d'interaction}) pendant un transfert \textit{routine}.



%Within this handover framework, using humanoid robot HRP-2Kai, we first tested and confirmed the feasibility of these models under the scenario where human and robot co-workers always use their right hand and left end-effector respectively. Later a generalized framework was presented and tested where both co-workers were able to choose either of their hand to handover or exchange the object. In addition, thanks to the native low-level controller of our robot HRP-2Kai and by utilizing whole-body configuration, we were able to further extend our handover framework, which allowed robot to use both hands  (bi-manual) simultaneously during the object handover \textit{routine}.

Dans ce framework de transfert, en utilisant le robot humanoïde HRP-2Kai, nous avons d'abord testé et confirmé la faisabilité de ces modèles dans le scénario où les collaborateurs humains et robotiques utilisent toujours leur effecteur droit et gauche respectivement. Par la suite, un framework généralisé a été présenté et testé où les deux collaborateurs ont pu choisir entre la remise ou l'échange de l'objet de leur main. De plus, grâce au contrôleur natif de bas niveau de notre robot HRP-2Kai et à l'utilisation de la configuration corps entier, nous avons pu étendre notre framework de transfert, ce qui a permis au robot d'utiliser les deux mains (bi-manuel) simultanément pendant le transfert des objets \textit{routine}.


%Furthermore for an proactive handover of an object between human and robot, we believed it was important to consider the possibility of robot taking a step to handover or exchange an object with the human co-worker, in scenarios where a short distance travel is required. Hence, we explored the full capabilities of biped humanoid robot and added a scenario where robot needs to proactively take few steps in order to handover or exchange the object between its human co-worker. This scenario had been implemented on HRP-2Kai and tested during both when human-robot dyad uses either single or both hands simultaneously. 

De plus, pour un transfert proactif d'un objet entre l'homme et le robot, nous avons pensé qu'il était important d'envisager la possibilité que le robot fasse un pas pour transférer ou échanger un objet avec le collègue humain, dans les scénarios où un déplacement sur une courte distance est nécessaire. Nous avons donc exploré toutes les capacités d'un robot humanoïde bipède et ajouté un scénario où le robot doit prendre quelques mesures proactives pour transférer ou échanger l'objet entre ses collègues humains. Ce scénario a été mis en œuvre sur HRP-2Kai et testé pendant les deux lorsque la dyade human-robot utilise une seule main ou les deux simultanément. 




\paragraph*{\LARGE {Conclusion \\}\\}

%To conclude this thesis, our achieved results contributed in the broad field of human robot interactions (specially with the humanoid robot) both at a safer distance and in close proximity, namely during a human-robot interaction (HRI) and physical human robot interaction (pHRI) respectively. The work done in this thesis was about the interactions between human and biped humanoid robot as `co-workers' in the industrial scenarios. We started with the non-physical human-robot interaction scenario based on a industrially inspired \textit{Pick-n-Place} task example and then advanced towards the physical human-robot interactions with an example of human humanoid robot dual arm bi-directional object handover.

Pour conclure cette thèse, les résultats que nous avons obtenus ont contribué dans le vaste domaine des interactions entre robots humains (en particulier avec le robot humanoïde) à une distance plus sûre et à proximité immédiate, notamment lors d'une interaction humain-robot (HRI) et d'une interaction physique robot humain (pHRI) respectivement. Le travail effectué dans cette thèse portait sur les interactions entre les robots humanoïdes humains et bipèdes en tant que "collègues de travail" dans les scénarios industriels. Nous avons commencé avec le scénario d'interaction homme-robot non physique basé sur un exemple de tâche d'inspiration industrielle \textit{Pick-n-Place}, puis nous avons avancé vers les interactions physiques homme-robot avec un exemple de transfert d'objet bidirectionnel de bras robotisé humanoïde humain.


 
%In Chapter~\ref{distinct motor contagion} and~\ref{more than just co-workers}, we examined an empirical repetitive industrial task in which a human participant and a humanoid robot co-worker near each other. We primarily chose cyclic and repetitive \textit{pick-n-place} task for the experiments as we wanted a task that is simple but rich and could represent several industrial co-worker scenarios. We found that this is one of the most common task in current industrial platforms where robots are often employed. Note that this study didn't consider effects of factors such age, physical or behavioral characteristics of the partner co-worker. They may have had indirectly affected these two motor contagions, perhaps its an interesting topic of discussion and need to be explored in future research.


Dans le chapitre~\ref{distinct motor contagion} and~\ref{more than just co-workers}, nous avons examiné une tâche industrielle empirique répétitive dans laquelle un participant humain et un robot humanoïde collaborent l'un près de l'autre. Nous avons principalement choisi la tâche cyclique et répétitive \textit{pick-n-place} pour les expériences car nous voulions une tâche qui soit simple mais riche et qui puisse représenter plusieurs scénarios de collègues industriels. Nous avons constaté qu'il s'agit d'une des tâches les plus courantes dans les plateformes industrielles actuelles où les robots sont souvent utilisés. Notez que cette étude n'a pas tenu compte des effets de facteurs tels que l'âge, les caractéristiques physiques ou comportementales du collègue de travail du partenaire. Ils ont peut-être indirectement affecté ces deux contagions motrices, c'est peut-être un sujet de discussion intéressant et il faudra l'explorer dans les recherches futures.



%Note that in Chapter~\ref{more than just co-workers}, we quantify the participants level of the robot exposure by a self perceived robot exposure score by themselves rather than the actual robot exposure. As an actual standard robot exposure questionnaire to measure this effect is currently absent and the development of one can be useful to understand how this effects, such as the one we highlight here, vary over time.

Notez que dans chapitre~\ref{more than just co-workers}, nous quantifions le niveau d'exposition du robot des participants par un score d'exposition du robot auto-perçu par eux-mêmes plutôt que par l'exposition réelle du robot. Comme il n'existe pas actuellement de questionnaire standard sur l'exposition des robots pour mesurer cet effet, il peut être utile d'en développer un pour comprendre comment ces effets, comme celui que nous soulignons ici, varient dans le temps.


%Overall the findings mentioned and discussed in Chapters~\ref{distinct motor contagion} and~\ref{more than just co-workers} highlights several new features of motor contagions, but also open new questions for future research. One can find these results useful for customizing the design of robot co-workers in industries and sports in future studies by moderating or exploiting these contagions. While if ethically allowed, these contagions could be utilized to improve co-workers performance speed and hence productivity.

Dans l'ensemble, les résultats mentionnés et discutés dans les chapitres~\ref{distinct motor contagion} et~\ref{more than just co-workers} mettent en évidence plusieurs nouvelles caractéristiques des contagions motrices, mais ouvrent aussi de nouvelles questions pour des recherches futures. On peut trouver ces résultats utiles pour personnaliser la conception de robots collaborateurs dans l'industrie et le sport dans des études futures en modérant ou en exploitant ces contagions. Si l'éthique le permet, ces contagions pourraient être utilisées pour améliorer la vitesse de travail des collègues et, par conséquent, leur productivité.



%Finally we conclude Chapter~\ref{handover chapter} with a preliminary testing of complete handover framework under above mentioned scenarios including locomotion. We confirm that our bi-directional object handover framework is intuitive and adaptable to several objects of distinguishable physical properties (shape, size and mass), including  industrial tools. It only needs the information of handover object's shape and size, though knowing mass of the object is not important in the beginning. We confirmed the feasibility of our handover framework under several objects with mass ranges from [$0.17$ to $1.1$] kg, during above mentioned scenarios. However, over the course of several handover \textit{routine} trials, we report that the calculated object mass is not accurate and has marginal error of $ \pm10 $\% compared to object's actual mass and needs further improvement on better estimation of the inertial forces at play or  better ways to remove offsets from force sensors. Although it didn't affect the optimal threshold which is important at the time of object handover from robot to human co-worker.

Enfin, nous concluons le chapitre~\ref{handover chapter} par un essai préliminaire du framework de transfert complet sous les scénarios mentionnés ci-dessus, y compris la locomotion. Nous confirmons que notre framework bidirectionnel de transfert d'objets est intuitif et adaptable à plusieurs objets aux propriétés physiques distinctes (forme, taille et masse), y compris les outils industriels. Il n'a besoin que des informations sur la forme et la taille de l'objet remis, bien que la connaissance de la masse de l'objet ne soit pas importante au début. Nous avons confirmé la faisabilité de notre framework de transfert sous plusieurs objets dont la masse varie de[$0,17 $ à $ 1,1 $] kg, selon les scénarios mentionnés ci-dessus. Cependant, au cours de plusieurs essais de transfert \textit{routine}, nous signalons que la masse calculée de l'objet n'est pas exacte et a une erreur marginale de $ \pm10 $\% par rapport à la masse réelle de l'objet et doit encore être améliorée pour mieux estimer les forces inertielles en jeu ou pour trouver de meilleurs moyens de supprimer les décalages des capteurs de force. Bien que cela n'ait pas affecté le seuil optimal qui est important au moment de la remise de l'objet du robot à son collègue humain.



%Though overall our method allow us to not stop the motion of end-effector and still being able to handover (both ways) the object, however if the handover occurs while both human and robot end-effectors are moving then this problem of handover would be extend to the problem of object collaboration and manipulation, which is already being studied extensively in our research group. Therefore we concentrated solely on the proactive bi-directional handover problem and hence decided to reduce the end-effectors velocity (${}^{ef}v\simeq0$) at the time of handover. 

Bien que dans l'ensemble notre méthode nous permette de ne pas arrêter le mouvement de l'effecteur final tout en étant capable de transférer l'objet (dans les deux sens), cependant, si le transfert a lieu alors que les effecteurs finaux humains et robotiques sont en mouvement, ce problème de transfert serait étendu au problème de la collaboration et de la manipulation des objets, qui est déjà largement étudié dans notre groupe de recherche. Nous nous sommes donc concentrés uniquement sur le problème de transfert bidirectionnel proactif et avons donc décidé de réduire la vitesse des effecteurs finaux (${}^{ef}v\simeq0$)  au moment du transfert. 


%By adding a dimension of step-walk locomotion, We did not focus on the problem of motion planning or navigation in a large cluttered environment but instead we concentrated our efforts to solve and optimize object handover problem which requires immediate shared efforts between human-robot dyad in a small space where few steps are necessary and enough for a comfortable and convenient object handover. Our proposed method is simple but effective to take advantage of biped humanoid robot and deal with the problem of bi-directional bi-manual object handover using robot whole-body control and locomotion.

En ajoutant une dimension de locomotion pas à pas, nous ne nous sommes pas concentrés sur le problème de la planification du mouvement ou de la navigation dans un environnement encombré, mais nous avons plutôt concentré nos efforts pour résoudre et optimiser le problème de transfert d'objets qui exige des efforts partagés immédiats entre la dyade homme-robot dans un petit espace où peu de pas sont nécessaires et suffisants pour un transfert confortable et pratique des objets. La méthode que nous proposons est simple mais efficace pour tirer profit des robots humanoïdes bipèdes et traiter le problème du transfert bidirectionnel bidirectionnel d'objets en utilisant le contrôle du corps entier et la locomotion du robot.


\clearpage % end of resume de la thesis
