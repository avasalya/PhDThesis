% ====================================================================== %
%                           State-of-the-art
% ======================================================================= %

{\color{blue}\chapter{State-of-the-art}\label{sota}}


\section{Humanoid Robots}
%\url{file:///home/avasalya/Dropbox%20(CNRS-AIST%20JRL)/PhD/Papers/other's%20thesis/thesis-antonio-bussy.pdf} 1.1.2 Humanoid Robots
%	
%overivew of humanoid technologies ~\cite{stasse2019overview}
%
%\url{file:///home/avasalya/Dropbox%20(CNRS-AIST%20JRL)/PhD/Papers/other's%20thesis/thesis-paul-evrard.pdf} 1.1.1 Humanoid robots


\subsection{Humanoid robot control}


\subsection{Stabilization and walking}


\clearpage
\section{Human robot interaction(HRI)}


%%2.1.6 Challenges for the application to pHRI: reasons for a haptic language
%%thesis paul-evrard
%
%
%\url{file:///home/avasalya/Dropbox%20(CNRS-AIST%20JRL)/PhD/Papers/other's%20thesis/thesis-antonio-bussy.pdf}
%	
%{\bf Human-in-the-loop // Collaborative Task // Haptic and Vision }
%	
%%\url{https://onedrive.live.com/redir?resid=52CDCC7290FBBB36%21414703&page=Edit&wd=target%28Read%20paper%20summary.one%7C664fe7ce-8808-4584-88a1-91adf3b4b827%2FHuman-in-the-loop%20%5C%2F%5C%2F%20Collaborative%20%7C6df6c193-c236-423d-94c7-1ae0fdda87a0%2F%29}

\subsection{Cognitive Robotics}
Human cognition is a field that deals with the study and research on how humans learn motor skills and preceptual behaviour using their sensory information. Cognitive Robotics on the other hand is a field of study based on the research on how robots can learn and acquire such human skills to carry to human-like tasks on a daily basis, either alone or during an active interaction with the humans.

%http://www.cogsci.rpi.edu/~heuveb/Teaching/CognitiveRobotics/What%20Is%20Cognitive%20Robotics.html\\
%
%However, this thesis partially focuses on the behavioural effect of those human skills learned humanoid robots on their human co-worker during a common task and rest of thesis deals with study of physical interaction between human and robot during an interactive task of object handover.

%thesis agravante
%1.1.1Terminology with a cognitive and behavioral science

\subsection{Behavioral science: motor contagion}
%\url{https://onedrive.live.com/redir?resid=52CDCC7290FBBB36%21414703&page=Edit&wd=target%28Read%20paper%20summary.one%7C664fe7ce-8808-4584-88a1-91adf3b4b827%2FSummary%7Cc9d7bb87-4a0b-4d20-8084-fbfa6f12f0fd%2F%29}


\clearpage
\section{Physical human robot interaction(pHRI)}

\subsection{Robot handover behaviors}
%\url{https://personalrobotics.cs.washington.edu/publications/strabala2013handoff.pdf}
%
%strabala paper important for handover introduction

\subsection{Previous handover studies}
%before collborative task in HRI comes handing over of an object, next section we discuss recent work in the object handover between human and robots


\subsection{Passive \& proactive approaches}


\subsection{Walking and handover}


\clearpage