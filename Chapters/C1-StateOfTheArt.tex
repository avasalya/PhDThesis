% ====================================================================== %
%                           State-of-the-art
% ======================================================================= %

{\color{blue}\chapter{State of the art}\label{sota}}

\section{Humanoid Robots}
%\url{file:///home/avasalya/Dropbox%20(CNRS-AIST%20JRL)/PhD/Papers/other's%20thesis/thesis-antonio-bussy.pdf} 1.1.2 Humanoid Robots
%	
%overivew of humanoid technologies ~\cite{stasse2019overview}
%
%\url{file:///home/avasalya/Dropbox%20(CNRS-AIST%20JRL)/PhD/Papers/other's%20thesis/thesis-paul-evrard.pdf} 1.1.1 Humanoid robots


\subsection{Humanoid robot control}
%\url{https://www.ieee-ras.org/whole-body-control}
%whole body control~\cite{sentis2006whole} A Whole-Body Control Framework for Humanoids Operating in Human Environments

\subsection{Biped locomotion}


%\clearpage

\section{Human robot interaction(HRI)}

%%2.1.6 Challenges for the application to pHRI: reasons for a haptic language
%%thesis paul-evrard
%
%
%\url{file:///home/avasalya/Dropbox%20(CNRS-AIST%20JRL)/PhD/Papers/other's%20thesis/thesis-antonio-bussy.pdf}
%	
%{\bf Human-in-the-loop // Collaborative Task // Haptic and Vision }
%	
%%\url{https://onedrive.live.com/redir?resid=52CDCC7290FBBB36%21414703&page=Edit&wd=target%28Read%20paper%20summary.one%7C664fe7ce-8808-4584-88a1-91adf3b4b827%2FHuman-in-the-loop%20%5C%2F%5C%2F%20Collaborative%20%7C6df6c193-c236-423d-94c7-1ae0fdda87a0%2F%29}

\subsection{Cognitive Robotics}
Human cognition is a field that deals with the study and research on how humans learn motor skills and preceptual behaviour using their sensory information. Cognitive Robotics on the other hand is a field of study based on the research on how robots can learn and acquire such human skills to carry to human-like tasks on a daily basis, either alone or during an active interaction with the humans.


\textbf{In this study we wanted to choose a task that is simple, yet rich, and is representative of many industrial co-worker scenarios. We found that (repetitive) pick and place tasks to be the most common industrial tasks in which robots are employed. We therefore chose to start with a cyclic touch task in this experiment.}


\paragraph{plosone}
\textbf{Our results show that human task frequency, but not task accuracy, is affected by the observation of a humanoid robot co-worker, provided the robot's head and torso are visible.
}

\paragraph{roman}
\textbf{ Our findings suggest that on-line contagions affect participant's movement frequency while the \textit{off-line} contagions mainly affect their movement velocity. Moreover, on-line contagions were equally effective with either a human or a humanoid robot co-worker, and the \textit{off-line} contagions were notable after observing another human. Finally these results suggest that the actions by a humanoid robot co-worker can induce distinct effects on human behaviors, during and after observation.}



%http://www.cogsci.rpi.edu/~heuveb/Teaching/CognitiveRobotics/What%20Is%20Cognitive%20Robotics.html\\
%
%However, this thesis partially focuses on the behavioural effect of those human skills learned humanoid robots on their human co-worker during a common task and rest of thesis deals with study of physical interaction between human and robot during an interactive task of object handover.

%thesis agravante
%1.1.1Terminology with a cognitive and behavioral science

\subsection{Behavioral science: motor contagion}\label{motor contagions}
%\url{https://onedrive.live.com/redir?resid=52CDCC7290FBBB36%21414703&page=Edit&wd=target%28Read%20paper%20summary.one%7C664fe7ce-8808-4584-88a1-91adf3b4b827%2FSummary%7Cc9d7bb87-4a0b-4d20-8084-fbfa6f12f0fd%2F%29}

Motor contagions are implicit effects that cause certain features of an individual's action (like kinematics, goal, or outcome) to become similar to that of the observed action. Studies over the past two decades have reported various motor contagions in human behaviors caused by the observation of other humans as well as robots \cite{Blakemore:Neuropsychologia:2005, Fadiga:JNeuroPhys:1995, Ganesh:Springer:2015, Sciutti:IJSR:2012, Prinz:EJPAP:1997}. Understanding the effects of robots has recently developed pace due to the increased use of robots in co-worker scenarios with humans. In these scenarios, understanding how the behavior of robots affect humans can be beneficial for developing robot behaviors, both to ensure that they are perceived well and do not disturb humans, as well as for modulating human behaviors for the benefit of the task and humans.  

Motor contagions may be divided into two categories depending on when, relative to the action observation, they are induced. \emph{On-line contagions} are induced \emph{during} the observation of actions performed by another human or robot~\cite{Kupferberg:Methods:2009, Oztop:RAS_ICHR:2004, Chaminade:JPP:2009, Kupferberg:PlosOne:2012, Brass:ActaPsych:2001, Press:CBR:2005}. For example,~\cite{Kilner:CurBio:2003} analyzed the variance in movements of a human participant when s/he observed spatially congruent and in-congruent movements made by either another human or a robot. Their experiment thus focused on on-line contagions, and showed that on-line contagions (in terms of a change in movement variance) are induced while observing a human but not while observing a robot making non-biological movements.

\emph{Off-line contagions} on the other hand, are effects induced \emph{after} the observation of actions by another human or robot~\cite{Noy:B&C:2009, Kilner:SocialNeuro:2007, Bisio:PlosOne:2010, heyes2011automatic, Ikegami:SciReport:2014, Ikegami:elife:2018}. For example,~\cite{Bisio:PlosOne:2014} measured changes in a participant's hand velocity, with and without an object, after observing the same movement being performed by a human or a humanoid robot. Their result shows that the observation of movement can subsequently affect a participant's hand velocity, both when the observed movements are by a human, or a robot, but again, only when the humanoid robot followed biological laws of motion.

Though both on-line and off-line contagions have been extensively investigated, all the previous studies have concentrated on either type of contagion and never analyzed the two together. Therefore, it remains unclear whether and how the on-line and off-line contagions are different in terms of the movement features they affect, and the magnitude of these effects. Here we address this question by comparing the on-line and off-line contagions induced in participants by the observation of the same actions, performed either by a human, or a humanoid robot.

We examined an empirical repetitive industrial task in which a human participant and a co-worker (either a humanoid robot or another human) work near each other. We systematically varied the behavior, specifically movement frequency, of the co-worker task and examined the on-line and off-line contagions that are induced. The induced contagions were examined when the robot made biological (or human type), and when it made non-biological (or industrial) movements. We specifically examined three questions:

\begin{enumerate}
	\item Can on-line and off-line contagions from the observation of a same movement affect different movement features of the human participant?
	\item how do the strengths of the on-line and off-line contagions vary with the nature of the co-worker (ie. if human or robot), and the behavior of the co-worker?
	\item Consequently, are the on-line and off-line contagions different, or do they constitute the same effect observed at different instances?
\end{enumerate}


\subsection{Motor contagion between human and robot}\label{influence performance}

Robotics is now increasingly shifting to service and application fields, where robots need to collaborate with, and work in close proximity to, human co-workers. In these scenarios, it is of prime importance to understand how the presence of a robot co-worker influences the performance of humans around them. This understanding is essential not just in regard to productivity, but also in order to monitor and control any emotional and motor effects the presence of robot co-workers may have on the humans.

Observation of actions performed by others is known to induce implicit effects on an individual's action. These effects, that are referred to as motor contagions, have been extensively studied in psychology and sports science~\cite{heyes2011automatic,Blakemore:Neuropsychologia:2005,Becchio:BJN:2007,Ganesh:Springer:2015,Ikegami:SciReport:2014,Hillebrandt:SciReports:2014,Chaminade:BRB:2008,Oztop:RAS_ICHR:2004,Kilner:CurBio:2003,Sciutti:IJSR:2012}. In comparison, studies of motor contagions during human-robot interactions~\cite{Vasalya:roman:2018} are sparse, and have examined either how the observation of robots affect a human's movement velocity~\cite{Noy:B&C:2009,Kilner:SocialNeuro:2007,Bisio:PlosOne:2010,Bisio:PlosOne:2014}, or how it affects a human's movement variance~\cite{Kupferberg:Methods:2009,Kupferberg:PlosOne:2012,Brass:ActaPsych:2001,Press:CBR:2005}. However, the studies that reported changes in movement variance utilize arguably abstract tasks, and the studies reporting changes in movement speed do not analyze at how the participant movement variance changed with the speed. On the other hand, most industrial tasks require specific precisions in the movements and therefore, the performance in these tasks needs to be defined by considering both task speed (or frequency) and task accuracy. Here primarily, we analyzed how looking at a robot affects both, the speed and variance of the observing human's movement, to see whether we can quantify how human \textit{performance} is affected by motor contagions.  
Furthermore, while there is contradictory evidence to suggest that the physical form of a robot co-worker (specifically whether it is humanoid or not) does~\cite{Chaminade:JPP:2009} or does not~\cite{Kupferberg:PlosOne:2012} affect the variance of movements by human's, it is unclear whether this is also true for the case of movements speeds, and hence performance. Finally, it is unclear whether and how the performance effects due to a robot co-worker are modulated by a human co-worker's prior experience with robots, an issue that is crucial to understand how the human performance will change with continued exposure to a robot co-worker.


\textbf{Studies in motor control have exhibited that human movements are constrained by motor noise, which increases with the magnitude of motor commands in the muscles~\cite{Harris:Nature:1998}. In the case of `regular' and automatic movements in daily life, this leads to a trade-off between the speed and accuracy of the movement~\cite{Fitts:JEP:1954}. However, the accuracy of movements is also modulated by the regulation of arm impedance by muscle co-contraction~\cite{Burdet:nature:2001, Franklin:JoN:2008, Ganesh:RAS:2013}. As mentioned earlier, to comment on the task performance of the human co-worker, we next analyzed whether and how the touch accuracy of the participants changed alongside the contagions in their {\it htp}. 
}

%\st{see Fig.~\ref{fig:setup}}
To address these issues, we examined an empirical repetitive industrial task in which a human participant and a humanoid robot work near each other. We systematically varied the behavior, specifically frequency of robot movements and examined whether and how the frequency of movements by the human participants, and their task accuracy, is affected by the presence of the robot. To investigate the effect of physical form, we added conditions where the robot co-worker torso and head were covered, and only the moving arm was visible to the human participants. Finally, in order to compare the humanoid co-worker to a human co-worker, we also checked how the effects on the participants changed with a human co-worker, with and without his/her torso and head visible. To anticipate our results, we found that the presence of a humanoid co-worker can affect human performance, but only when it's humanoid form is visible. Furthermore, the effect was observed to increase with prior robot experience by the humans.


%\clearpage

\section{Physical human robot interaction(pHRI)}

\subsection{Proactive approaches in pHRI}


\subsection{Robot handover behaviors}
%\url{https://personalrobotics.cs.washington.edu/publications/strabala2013handoff.pdf}
%
%strabala paper important for handover introduction

The usage of robots in the personal as well as in the commercial sectors have evolved significantly in the recent years. Moreover, robot working and sharing common workspace with humans as co-workers in these sectors can often lead to opportunities where human and robot have to work together and collaborate within a confined space. One of the most often task that occurs during these human-robot collaborative interaction is the handing-over or exchange of an object such as tools in industrial scenarios or a glass of water in personal scenarios from either robot to human or vice-versa. This problem of object handover is a complex collaborative task that occurs seamlessly and effortlessly during the physical interaction between the human dyads, often without any explicit communication.  Object handover examples such as handing over a glass of water to a patient by the care-giver, sharing a tool to a mechanic, handing business card to a client and many more are often fundamental in our society. These natural yet simple physical interactive task occurs flawlessly multiple times between the human dyads on a daily basis and under several scenarios in space and time. Although handovers are fluent phase-less natural events between human-human interactions, but during the human-robot dyad interactions, the handover of an object is a challenging task and often regarded as unnatural behavior. 

This unnatural behavior mainly arises due the lack of responsiveness and unreliability of the robot co-worker, and the safety issues of the human co-worker during the interaction. In the previous human-robot interaction studies~\cite{huber2008human, strabala2013toward, shibata1995experimental} and also in our work Chapter (\nameref{more than just co-workers}), we have shown that the human acceptance of the robot co-worker during a task increases when the robot appears and behaves human-like, specially during a collaborative interactive task. Therefore, in our study we primarily chose to consider humanoid robot HRP-2Kai as the robot co-worker. In this Chapter we particularly focused on solving the problem of intuitive and proactive bi-directional object handover between human and a biped humanoid robot dyad using robot whole-body control (WBC). We take inspiration and insights from the previous works in the field of object handover between human-human and human-robot dyads in general and formulate our handover problem.



\subsubsection{Previous handover studies}
%\subsection{Related work}

%before collborative task in HRI comes handing over of an object, next section we discuss recent work in the object handover between human and robots

Object handover being the most common interactive task and knowing its significance in daily life, it is obvious that object handovers have been widely studied by the researchers both during the interactions of human-human and human-robot dyads. These past studies related to object handovers can be categorized under three main research topics. The prediction and estimation of human motion towards the handover location and concurrent robot motion planning towards that location~\cite{huber2008Indus, li2015predicting, waldhart2015planning, mainprice2012sharing, vahrenkamp2009humanoid, kim2004advanced, mainprice2010planning}. To understand and codify the interaction forces that being applied on the object between the dyads during the handover~\cite{chan2014implementation, medina2016human, chan2013human, sadigh2009safe, nagata1998delivery}. To effectively minimizing the overall handover duration between the dyads~\cite{nemlekarprompt, cakmak2011using, huber2008human, nemlekar2019object}.


\textbf{Studies on handover motion\\}
In order for the human-robot object handovers to be proactively smooth and intuitive, it is necessary for the robot to be able to predict and estimate the human motion in advance. Instead of simply waiting for the object to be presented by the human at the handover location, the robot must proactively plan its own motion by observing and predicting next human motion and arrive at the human chosen handover location approximately at same time. ~\cite{li2015predicting} have compared several mathematical position prediction models where human is always \textit{giver} and robot is always \textit{receiver}. Thanks to these models, their preliminary results show overall reduction in handover duration period. While~\cite{perez2015fast, sheikholeslami2018prediction, vogt2017system, kupcsik2016learning}  have developed human hand motion prediction model by gaining insights learned from the human-human object handover studies. However, since specifically developing alone position prediction model is not the main focus of our study, therefore we primarily decided to adapt a common method to predict the human hand motion using~\textit{constant velocity} based model. We have further discussed this in detail in the Chapter (\nameref{handover chapter}).

\textbf{Studies on handover configuration\\}
Although predicting handover location is not enough, robot must also be able to find the most appropriate configuration to grasp (as \textit{receiver}) or release (as \textit{giver}) the object, based on the comfort and requirement of the human co-worker. Therefore for an intuitive and smooth handover of an object, the robot should relatively orient it's hand (in our case gripper as the end-effector) and configure accordingly. Though there are several possible configurations to handover an object, but it is crucial for the robot to determine the correct configuration of its end-effector during the handover which is suitable and natural for the human. Moreover, according to~\cite{cakmak2011human}, we humans prefer handover of an object in its default orientation.~\cite{aleotti2012comfortable} work suggests that robot takes human comfort and convenience under consideration in order to find the appropriate orientation during the handover, which they gave higher score based on the appropriateness and safety.~\cite{vogt2018one} showcased human-robot object handover using imitation learning based on human-human demonstration. The robot relies on the posture of human participant to determine the pose of object handover. ~\cite{kim2004advanced} proposed object handover and grasp planning method that incorporates cultural etiquette (one-hand, two-hand, two-hand mid-air) based on the object's function, object shape and safety of both human and robot. While~\cite{vezzani2017novel, song2013predicting, micelli2011perception} estimated the appropriate handover orientation using 3D image of the object and by tracking human hand.~\cite{lopez2006grasp} introduced a simulated model planner to grasp the unknown objects during the interactive manipulative tasks. However even though the handover motion can be planned in advance or optimized online but it is still essential for the robot to be promptly adapt during the interaction, specially at the time of handover or exchange of the object.

\textbf{Studies on interaction forces\\}
Previous studies on the forces during these handover interactions analyzed the relationship between grip force (applied on the object by the human) and the load force (object weight shared by the robot).~\cite{mason2005grip} findings suggest a gradual change in the grip force while the human dyad implicitly share the load force during the transfer of the object.~\cite{chan2013human} investigated the grip forces applied on an object while it is being exchanged between the human-human dyad during a handover. They found a linear relationship between these forces, suggesting that the \textit{giver} is responsible for safety of object during the transfer and \textit{receiver} is responsible for the timing of handover.~\cite{nagata1998delivery} demonstrated a grasping system based on the 6DOF wrench (force and torque) feedback that first acknowledges the stable and secure grasp on the object by the human, only after that the object is released by the robot.~\cite{medina2016human} learned from the insights during the human dyad handovers and developed a dynamic force controller which greatly reduces the internal forces between the human-robot dyad compared to the traditional threshold based controller. Inspired by the human object grasping~\cite{sadigh2009safe} designed a robotic grasping controller with minimal normal forces while grasping an object to make sure it does not slip.~\cite{parastegari2018failure} proposed object re-grasping controller in case of false grasping.~\cite{kupcsik2016learning} presented a dynamic object handover controller based on the contextual policy search, where robot learns about the object handover while interacting with the human and dynamically adapts to the motion of the human.~\cite{chan2014implementation} designed a controller to estimate the applied grip force and load force by measuring the joint position/angle errors on a compliant underactuated humanoid hand. However in most of these studies, knowledge of object mass is a prerequisite. In our approach, knowing object mass in advance is optional as it can be calculated during the handover routine but we do rely on knowing the object physical structural properties. 

\subsubsection{Locomotion and handover}

\textbf{Studies on robot motion planning\\}
Majority of studies related to human-robot object handovers were carried out in the past with traditional robotic arm manipulators attached to either a stationary-base or to a wheeled-base mobile robotic system~\cite{medina2016human, vogt2018one, huber2008Indus, kupcsik2016learning, cakmak2011human} and latter studies are often related to the robot motion planning and navigation in a large space and lacks proactive behavior using biped locomotion of a humanoid robot which are capable of walking like human does. To the best of our knowledge, bi-directional object handover with biped walking have not been considered in the previous works on the human-robot dyad object handover. However~\cite{vezzani2017novel, chan2014implementation} have utilized biped walking capable humanoid robots in their studies but without considering locomotion. Therefore in order for the robot to be fully proactive, we believe it is important to consider the possibility of robot taking a step to handover or exchange an object with the human co-worker, in scenarios where a short distance travel is required. We do not focus on the problem of motion planning in a large cluttered environment~\cite{mainprice2012sharing, vahrenkamp2009humanoid, kim2004advanced} but instead we concentrated our efforts to solve and optimize object handover problem which requires immediate shared efforts between human-robot dyad in a small space where a \textit{one-step-walk} cycle is necessary and enough for a comfortable and convenient object handover. We proposed simple but effective methods to take advantage of humanoid robot and deal with the problem of bi-directional object handover using robot whole-body control.

\clearpage
