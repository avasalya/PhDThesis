{\color{blue}\chapter*{Abstract}}
\addcontentsline{toc}{chapter}{Abstract}
\thispagestyle{empty}

%\noindent \hrule\vspace{2pt}
%\par\nobreak
%\noindent\textbf{\textsc{Title: Human and humanoid robot co-workers: motor contagions and whole-body handover}}
%\noindent \vspace{2pt}\hrule\vspace{2pt}

%%% link

The work done in this thesis is about the interactions between human and humanoid robot HRP-2Kai as co-workers in the industrial scenarios. By interactions, we started with the non-physical human-robot interaction scenario based on a industrially inspired \textit{Pick-n-Place} task example and then advanced towards the physical human-robot interactions with an example of human humanoid robot dual arm bi-directional object handover. The research topics in thesis are divided into two categories. In the context of \textit{non-physical} human-robot interactions, the studies conducted in the 1$^{st}$ part of this thesis are mostly motivated by social interactions between human and humanoid robot co-workers, which deal with the implicit behavioral and cognitive aspects of interactions. While in the context of \textit{physical} human-robot interactions, the 2$^{nd}$ part of this thesis is motivated by the physical manipulations during object handover between human and humanoid robot co-workers in close proximity using humanoid robot whole-body control framework and locomotion.


%%% roman
When an individual (human and robot) performs an action followed by the observation of someone's action, behavioral implicit effects such as motor contagions causes certain features (kinematics parameters, goal or outcome) of that action to become similar to the observed action. However previous studies have examined the effects of motor contagions induced either during the observation of action or after but never together therefore it remains unclear weather and how these effects are distinct from each other.

We designed a paradigm and a repetitive task inspired by the industrial \textit{Pick-n-Place} movement task, in first HRI study, we examine the effect of motor contagions induced in participants during (we call it \textit{on-line} contagions) and after (\textit{off-line} contagions) the observation of the same movements performed by a human, or a humanoid robot co-worker.

The results from this study have suggested that \textit{off-line} contagions affects participant's movement velocity while \textit{on-line} contagions affect their movement frequency. Interestingly, our findings suggest that the nature of the co-worker, (human or a robot), \textit{tend} to influence the \textit{off-line} contagions significantly more than the \textit{on-line} contagions.


%%% plosone
Moreover, while the past studies have examined how the effects of induced motor contagions due to the observation of human and robot movements have affected either human co-worker's movement velocity or how it affected movement variance but never both together. Therefore we argue that since precision in movements along with speed is the key in most industrial tasks, hence it is necessary to consider both task accuracy and task speed to accurately measure the performance in a task.

Therefore in second HRI study, under the same paradigm and repetitive industrial task, we systematically varied the robot behavior, and observed whether and how the performance of a human participant is affected by the presence of humanoid robot. We also investigated the effect of physical form of humanoid robot co-worker where torso and head were covered, and only the moving arm was visible to the human participants. Later, we compared these behaviors with a human co-worker, and examined how the observed behavioral affects scale with experience of robots. 

Our results show that the human and humanoid robot co-workers have been able to affect the performance frequencies of the participants, while their task accuracy remained undisturbed and unaffected. However with the robot co-worker, this is true only when the robot head and torso were visible and robot made biological movements.


%%% handover
Next, in pHRI study, we designed an intuitive bi-directional object handover routine between human and biped humanoid robot co-worker using whole-body control and locomotion, we designed models to predict and estimate the handover position in advance along with estimating the grasp configuration of object and active human hand during handover trials. We also designed a model to minimize the interaction forces during the handover of an unknown mass object along with the timing of object handover routine.

We mainly focused on three important key features during the human humanoid robot object handover routine ---the \textit{timing}(s) of handover, the \textit{pose} of handover and the \textit{magnitude} of the interaction forces between human hand(s) and humanoid robot end-effector(s). Basically we answer the following questions, ---\textbf{when}(\textit{timing}), \textbf{where} (\textit{position in space}), \textbf{how}(\textit{orientation and interaction forces}) of the handover.

Later, we present generalized handover controller, where both human and robot is capable of selecting either of their hand to handover and exchange the object. Furthermore, by utilizing whole-body control configuration, our handover controller is able to allow robot to use both hands simultaneously during the object handover. Depending upon the shape and size of the object that needs to be transferred.

Finally, we explored the full capabilities of biped humanoid robot and added a scenario where robot needs to proactively take few steps in order to handover or exchange the object between its human co-worker. We have tested this scenario on real humanoid robot HRP-2Kai during both when human-robot dyad uses either single or both hands simultaneously.











\vspace{3pt}\hrule\vspace{3pt}
\noindent \textbf{\textsc{Keywords:}}
Robotics, Humanoid robot co-worker, Behavior, Robotic behavior, Human performance, Motor contagion, Human-robot interaction, Physical human-robot interaction
\noindent \vspace{3pt}\hrule\vspace{3pt}
\noindent \textbf{\textsc{Discipline~:}}
Syst\`emes Avanc\'es et Micro\'electronique
\noindent \vspace{3pt}\hrule\vspace{3pt}
\noindent Laboratoire d'Informatique, de Robotique et de Micro\'electronique de Montpellier\\
UMR 5506 CNRS/Universit\'e de Montpellier\\
B\^atiment 5 - 860 rue de Saint Priest
