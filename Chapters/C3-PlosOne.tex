% ======================================================================= %
                %Motor Contagions Influence Performance
% ======================================================================= %

{\color{blue}\chapter{More than just co-workers}\label{more than just co-workers}}

\section{Introduction}

In the previous Chapter, we explored distinct motor contagions induced in the human co-worker during (\textit{on-line}) and after (\textit{off-line}) the observation of an action by either the human or robot co-worker. Here, under the same experimental setup and the task, we focused on exploiting the induced motor contagions in the human co-worker. We primarily asked one specific question in this Chapter ---Does the presence of a humanoid robot co-worker influence the performance of humans around it? 

During the human-robot interactions, previous studies on motor contagions have either analyzed the affects of robot co-worker movement observation on human's movement velocity or the affects on their movement variance, but never both together. We argue here that the performance has to be measured considering together, both the task speed (or frequency) as well as task accuracy. Therefore in the same experimental task as discussed in the previous Chapter section~\ref{roman method}, along with the addition of three new conditions, by coherently varying the robot behavior, we observed how the performance of a human participant co-worker is affected by the presence of the humanoid robot co-worker. Where in two new conditions, we investigated the effect of physical form, by covering the head and torso of the human and robot co-workers in the human covered and robot covered respective conditions and only the moving robot arm was visible to the participants. While in the third new condition (robot non-biol), we investigated the effects of robot movement velocity profile with the velocity-phases of both industrial and biological trajectories. Finally, as ground truth, we compared these behaviors with a human co-worker, and also examined how this observed behavioral affects vary on the scale with the experience of robots. 

%\clearpage

\section{Materials and methods} \label{methods}

\begin{figure}[b]
	\centering{\includegraphics[width=1\columnwidth]{plots/c3-plots/Fig1}}
	\caption{{\bf Experimental setup.} The participants in our experiment worked in six conditions; with a robot performing \textit{biological} movements in A) robot co-worker condition; B) human co-worker condition; to check relevance of  human form in C) robot covered co-worker condition; and D) human covered co-worker condition; E) a robot co-worker performing \textit{non-biological} movements in robot non-biol co-worker condition; F) a robot co-worker performing industrial movements. The coordinate axis defining the movement setup is indicated in white (A).}
	\label{fig:setup1}
\end{figure}

The arrangement of experimental setup and task are same as mentioned in the Chapter~\ref{distinct motor contagion}. However the total number of conditions and features analyzed are different here. To explain clearly and avoid confusion, we have renamed the conditions as can be seen in Fig.~\ref{fig:setup1} and mentioned again some important details in the below Subsection.

\subsection{Experimental task and conditions}

As mentioned in previous Chapter, our task was motivated by the industrial \textit{pick-n-place} part assembly task, which required the participants to touch repeatedly inside two static red circles on the touchscreen with a hand stylus. The participants were only instructed to touch inside each circle consecutively, however they were free to touch anywhere inside each circle. During a preliminary experiment, we discover that the participant's touches had less than 1 cm standard deviation, both in the $x$ and $y$ movement directions. Therefore, on purpose we chose to keep the radius of the touch circles more than \texttt{2x} times larger than their standard deviation in the main experiment. Hence the target size of touch circles were 5 cm in diameter. This increase in the size of touch circles were crucial to us since the participants were asked to touch the circle in each trial, but they were free to touch anywhere on the circle. This enabled us to observe and measure the variance along with standard deviation in participant's touch position (that may result contagion in their speed) across our experiment.
 
A co-worker being the human experimenter or HRP-2Kai humanoid robot worked on the same task in front of the participants. Again the participants were asked to perform their task at their own chosen `comfortable' frequency, and ignore the co-worker. The trial protocol was same as mentioned in the previous Chapter. The participants worked in a series of 50 second trials with the co-worker. In a trial, participants initially performed alone for 10 seconds (participant-alone period), performed with the co-worker for next 20 seconds (together period) and then relaxed while watching the co-worker performs the task for the last 20 seconds (co-worker-alone period) (Fig.~\ref{fig:trialprotocol2}).   

\begin{figure}[hpt]
	\centering{\includegraphics[width=1\columnwidth]{plots/c3-plots/Fig2}}
	\caption{{\bf Trial protocol.} The participants worked in repeated trials with either a robot or human co-worker (the figure shows the trial with a robot co-worker). Each trial consisted of period when the participant worked alone and co-worker relaxed (participant-alone period), both worked together (together period), and the co-worker worked alone (co-worker-alone period). The notation of the time variable (represented in general by $\tau$) in each period are shown.}
	\label{fig:trialprotocol2}
\end{figure}

All participants wore ear buds and headphones, which enabled them to hear only white noise and had no external audio feedback, it was later confirmed in the post experiment questionnaire, Q6. All participants were instructed to ``{\it always hold the stylus like a stamp and touch alternatively inside each red circle on the touchscreen with continuous and smooth hand movements at a comfortable speed}''. While they were specifically told to ``{\it focus on your own task and ignore the co-worker when he/it starts after them}''. No other instructions were given regarding their hand speed and trajectory.

Unlike Chapter~\ref{distinct motor contagion}, here we have studied and analyzed data from all six experimental conditions. In four conditions, the participants worked with a HRP-2Kai humanoid robot co-worker, specifically in (a) \textit{robot co-worker} in which the robot played back biological movements, and the whole robot was visible to the participant (b) \textit{robot covered co-worker}, in which the robot played back biological movements, but its head and torso were covered, and the participant could only see the robot's moving arm. (c) \textit{robot non-biol co-worker}, in which a fully visible robot performed \textit{non-biological} arm movements, (d) \textit{robot indus co-worker}, in which a fully visible robot performed \textit{industrial} arm movements. While in the remaining two conditions, participants worked with a trained human experimenter in, (e) \textit{human co-worker} and (f) \textit{human covered co-worker}, where the head and torso of the human experimenter were covered (see Fig.~\ref{fig:setup1}).

In the experiment, all participants worked over 10 trials in each condition and there were 3 conditions per experiment. We had 6 \emph{condition combination} groups, see Table~\ref{groupTable} and each participant was assigned to one of that group. Each participant worked in the robot co-worker condition (as main condition), and in addition to two out of five other remaining conditions. The order of the conditions was balanced across the combination groups. This allowed us to compare the behavior of the same participants in each condition in a combination group, with their behavior in the robot co-worker condition.


\begin{table}[hpt]
	\caption{condition combination groups (G)}
	{HRP-2Kai in robot co-worker (RV), robot covered co-worker (RC), robot non-biol co-worker (RN), robot indus co-worker (RI) conditions and Human experimenter in human co-worker (HV), human covered co-worker (HC) conditions. The order of conditions in a combination group were randomized across participants.}
	\label{groupTable}
	\begin{center}
		\begin{tabular}{|c cccccc|}
			\hline  
			{\bf Sessions/Groups} &  {\bf G1} &  {\bf G2} &  {\bf G3} &  {\bf G4} &  {\bf G5} &  {\bf G6}  \\ 
			\hline
			Session 1 & RI & RC & HV & RC & RN &  RC \\ 
			
			Session 2 & RV & RI & RV & HC & HV &  RN \\ 
			
			Session 3 & RN & RV & HC & RV & RV &  RV \\ 
			\hline 			
		\end{tabular} 
	\end{center}
\end{table}


In the robot co-worker conditions, robot movements were a playback of the human recorded movements of a previous human volunteer (see subsection~\nameref{HRP2_Traj} for details). In this study, we quantified the participant performance in the trials by their half time periods or {\it htp} (the average time between two consecutive alternate touches, measured using motion capture system), and the variance of their press location (measured as a change of mean and standard deviation of their touchscreen presses in the \texttt{X}-\texttt{Y} plane).

\subsection{HRP-2Kai movement trajectories} \label{HRP2_Traj}

The biological movements that were played on HRP-2Kai in robot co-worker and robot covered co-worker conditions were a playback of the human arm movements (Fig.~\ref{fig:trajectories}, blue plot). We have already discussed this in detail in the subsection of previous Chapter~\ref{hrpTraj}.

 
\begin{figure}[hpt]
		\centering{\includegraphics[width=0.8\columnwidth]{plots/c3-plots/Fig3}}
	\caption{{\bf HRP-2Kai movement trajectories.} The trajectories played by the robot in robot co-worker, robot covered co-worker, robot non-biol co-worker and robot indus co-worker conditions.}
	\label{fig:trajectories}
\end{figure}


It is well known that human movements can be characterized by a bell-shaped velocity profile~\cite{flash1985coordination}. Generally when the movements are governed goal oriented, (when the end positions of the goal is fixed). In such cases, it is possible to see the peak of the bell-shaped profile to be shifted forward (in the direction of goal target) in time when the precision is required near the end of motion. Similar case is also true in our task where the participants were required to touch inside circles, within a given target region. However the velocity profiles of human movements are normally characterized by a single peak. Therefore, to formulate and design a `non-biological' movement profile for the robot non-biol co-worker condition, which consists of essence from both biological and industrial movements, we developed a a new movement profile with multiple velocity peaks. This movement profile was developed in position-time (cyan plots in Fig.~\ref{fig:trajectories}) profile using segments of fifth and third order polynomials during the lift-off, carry, set-down phases~\cite{Biagiotti:Springer:2008} (see cyan plots in Fig.~\ref{fig:trajectories2}). Again, our observations suggested that human volunteers made movements predominantly in the \texttt{Y}-\texttt{Z} plane. Therefore, this piece-wise polynomial trajectory for the robot non-biol co-worker condition was also designed over the $y$ (horizontal) and $z$ (vertical) dimensions, while $x$ was always kept at constant zero.


\begin{figure}[hpt]
		\centering{\includegraphics[width=0.8\columnwidth]{plots/c3-plots/Fig4}}
	\caption{{\bf HRP-2Kai trajectory generation.} The time trajectories in the Y and Z axis by the HRP-2Kai in the robot non-biol co-worker and robot indus co-worker condition, and the via-points (blue circles) used to generate both trajectories.}
	\label{fig:trajectories2}
\end{figure}


The industrial trajectory was characterized by a constant velocity phase. Note that this trajectory is same as the trajectory in $R_{\text{nonbiol}}$ condition, which was mentioned in the previous Chapter. We have only changed name of the trajectory to robot indus co-worker (magenta plots in Figs.~\ref{fig:trajectories} and~\ref{fig:trajectories2}), again to avoid confusion with the results of previous Chapter.
 
%\clearpage

\section{Data analysis} \label{data_analysis}

\subsection{Variables}

Here again, our analysis is based on the position data of both the participant's and co-worker's stylus markers. Specifically here we analyzed the `time' behavioral variable across each movement between the red circles on the touchscreen. In this study, we were interested in the task performance of participants, and therefore we primarily concentrated on the `time' between alternate touches in each \textit{iteration} (a movement between two consecutive alternate touches), which we referred as the half-time period ({\it htp}) or $\tau$, and the location of their touches on the touchscreen (in the \texttt{X}-\texttt{Y} plane). It is worth mentioning again that the both participants and co-workers were instructed to make non-stop continuous movements between touches at their comfortable speed, and therefore we were able to tease out the individual \textit{iteration}s of participant's and co-worker's by focusing on the changes in the direction of $\texttt{y}$-velocity in their recorded motion capture data. We have already analyzed and discussed the various measures of kinematic parameters such as position, velocity and acceleration along the \texttt{Y} (horizontal) and \texttt{Z} (vertical) axes over each iteration in the Chapter~\ref{distinct motor contagion}.


\begin{figure}[hbpt]
	\centering{\includegraphics[width=1\columnwidth]{plots/c3-plots/Fig5}}
	\caption{{\bf Examples of linear regression  fits.} The change of participant's {\it htp} (the average time between two consecutive alternate touches) between the together period and alone-period ($\tau_p^t(i)-\tau_p^a(i)$), relative to the {\it htp} of the co-worker behavior in the same trial ($\tau_c (i)-Av(\tau_p^a)$), where $Av(\tau_p^a)$ represents the average undisturbed {\it htp} by a participant across all his/her participant-alone periods. Note that the (robot or human) co-worker {\it htp} was random across trials, and the data in plots here are the ensemble of the participant behaviors arranged in increasing co-worker's {\it htp} on the abscissa. Each plot represent a condition, A) robot co-worker (blue); B) human co-worker (orange); C) robot covered co-worker (dark blue); D) human covered co-worker (dark orange); E) robot non-biol co-worker (cyan); F) robot indus co-worker (magenta) conditions. We used the AIC to choose either a first or second order model to fit the data for each participant. The lines represent the tangent slopes at the minimal data abscissa value.}
	\label{fig:allfit}
\end{figure}


We quantified the motor contagion in a participant' {\it htp} (the average time between two consecutive alternate touches) by analyzing the change of participant's {\it htp} between the together period and alone-period (see Fig.~\ref{fig:trialprotocol2}) in a trial ($\tau_p^t(i)-\tau_p^a(i)$), relative to the {\it htp} of the co-worker behavior in the same trial ($\tau_c (i)-Av(\tau_p^a)$), where $Av(\tau_p^a)$ represents the average undisturbed {\it htp} by a participants across his/her participant-alone periods. We later regressed the data obtained on each participant with either a first or second order regression model. The regression model was chosen based on the Akaike Information Criteria (AIC)~\cite{Akaike:ISIT:1973} and using {\tt fitlm} function of \texttt{MATLAB}. Across participants, we collected the slope of tangent at the minimum abscissa value of data ($\min[\tau_c(i)-Av(\tau_p^a)]$). This slopes were then checked for normality using Shapiro-Wilk test. Finally, data was analyzed for difference from zero using a one sample T-test in case of normal distribution otherwise using a Signed Rank test. Fig.~\ref{fig:allfit}, shows the one sample fitting~\footnote{see Appendix~\ref{contagion appendix} for all participants \textit{htp} fitting figures in all conditions.} of \textit{htp} of randomly chosen participant in each of the six reported condition and the collection of slopes comparison across six conditions are plotted in Fig.~\ref{fig:slope_allcond}. We applied a similar procedure to analyze the change in average \texttt{X} and average \texttt{Y} press locations of participants touches, along with the standard deviation of \texttt{X} and \texttt{Y} press locations relative to the {\it htp} of the co-worker behavior in the same trial ($\tau_c(i)-Av(\tau_p^a)$). Fig.~\ref{fig:variance} shows the analysis of these slopes.

Next, we examine the significance of the human form, we further extended this study by adding two more conditions, specifically robot covered co-worker and human covered co-worker conditions, where we completely covered (hidden) the head and torso of the co-worker but only the moving arm was visible to the participants (see Fig.~\ref{fig:setup1}C, D or inset photos in Fig.~\ref{fig:slope_allcond}). While we kept the experimental settings and analysis in these two new conditions exactly same as in the robot co-worker and human co-worker conditions.


\begin{figure}[hpbt]
		\centering{\includegraphics[width=1\columnwidth]{plots/c3-plots/Fig6}}
	\caption{{\bf All six conditions \textit{htp} comparision.} The plot of the collection of slopes which is obtained in Fig.~\ref{fig:allfit} and ~\ref{S1_Fig} to~\ref{S6_Fig} supplementary figures. The condition-wise comparison of the change of participants {\it htp} with co-worker {\it htp}. P-values are Bonferroni corrected where required. The tangent slope at the minimum data abscissa value ($\min[\tau_c(i)-Av(\tau_p^a)]$) was collected across participants (as shown in Fig.~\ref{fig:allfit}), checked for normality using the Shapiro-Wilk test and then analyzed for difference from zero using a one sample T-test (in case the distribution was normal) or a Signed Rank test.}
	\label{fig:slope_allcond}
\end{figure}


\begin{figure}[hpbt]
	\centering{\includegraphics[width=1\columnwidth]{plots/c3-plots/Fig7}}
	\caption{{\bf Participants touch variance.} Change of participant touch position with A) robot co-worker {\it htp}; B) human co-worker {\it htp}. A similar procedure which was used to quantify {\it htp} was also used here (see subsection~\nameref{data_analysis}) to analyze the change in a participant's average X press location, average Y press location, standard deviation of X press location, and standard deviation of Y press locations relative to the {\it htp} of the co-worker behavior in the same trial.}
	\label{fig:variance}
\end{figure}


\subsection{Participant sample size}

As mentioned earlier in the subsection~\ref{roman sample size} of previous Chapter, that we initially recruited 35 participants, all them participated in the main robot co-worker condition and two out of remaining five other conditions, to enable a intra-participant one sample T-test between the robot co-worker and each of the remaining conditions. These five remaining conditions namely are human co-worker, robot covered co-worker, human covered co-worker, robot non-biol co-worker and robot indus co-worker. Thus giving us five \textit{participant group}, such that randomly selected 14 participants performed in robot co-worker condition along with one of the five other condition. The number `14' also represents the participant numbers in similar previous studies~\cite{Bisio:PlosOne:2010, Bisio:PlosOne:2014} and this participant number `14' also corresponds to the G* power analysis~\cite{Erdfelder:JBRMIC:1996} using two-way one sample {\it T}-test ($\alpha$ = 0.05, $\beta$ = 0.85, $d$ = 0.9)~\cite{Verma:power_analysis:2017} for the biological experiments. However, with these participant numbers, we found that the slopes in the same robot co-worker condition were not similar among the participant groups ($p<0.05$, one-way ANOVA). The {\it htp} slopes of robot co-worker condition were significantly different from zero with two participant groups (p=0.022, and p=0.038), marginally significant in two participant group (p=0.07, and p=0.08) and not significant in one participant group (p=0.36). As a majority of the values tended to be significant, hence we later proposed and decided to add 7 participants (50\%) across these groups (robot covered co-worker, robot non-biol co-worker and robot indus co-worker conditions), making a total of 42 participants. This addition of participants ensured that the {\it htp} slopes across the participant groups become similar ($P = 0.99$; one-way Kruskal-Wallis H-test). After removal of three outliers, this gave us participants numbers of 13 (human co-worker condition), 13 (human covered co-worker), 17 (robot covered co-worker), 17 (robot non-biol co-worker), 18 (robot indus co-worker), and 39 in total for the robot co-worker condition, see Table~\ref{sizeTable}.


\begin{table}[hpt]
	\caption{Participant sample size}
	\label{sizeTable}
	\begin{center}
		\begin{tabular}{|c c|}
			\hline  
			{\bf condition} &  {\bf sample size} \\ 
			\hline
			robot co-worker & 39 \\ 
			\hline
			human co-worker & 13 \\
			\hline
			robot covered co-worker & 17 \\
			\hline
			human covered co-worker & 13 \\
			\hline
			robot non-biol co-worker & 17 \\
			\hline
			robot indus co-worker & 18 \\
			\hline 			
		\end{tabular} 
	\end{center}
\end{table}


\subsection{Questionnaire}

\subsubsection{Perception and fatigue}

All participants in our experiment had to answer a short six-question post experiment questionnaire. A 7 point Likert scale was used to measure the participants response on each of these questions. Participants were requested to choose a score between 0 to 7, where 0 (Not at all), 7 (very strongly). These questionnaire was presented after every session they participated in:

\begin{enumerate}[start=1,label={Q\arabic*.}]
	\item My movements were affected when the agent was working with me.
	\item My movement speed was changed when the agent was working with me.
	\item I was tired during the experiment.
	\item I could maintain the movement speed that I wanted even when the robot was performing its task.
	\item I found it difficult to do my task when the agent was working with me.
	\item I could hear noises from the co-worker during the experiment.
\end{enumerate}

Q1, Q2, Q4 and Q5 were designed to access whether the participants cognitively realized the affects on their behavior due to the co-worker. A score close to one in Q1, Q2 and Q5 (and a score close to 7 in Q4) indicates that they did not consciously realize the effects. Therefore we considered the Q4 scores by subtracting the reported values from 7.

\subsubsection{Robot exposure questionnaire}

Soon after the end of our data collection, we noted that it is important to measure the participant's robot experienced and exposure to robots. Therefore further questionnaire of four questions were sent to participants:

\begin{enumerate}[start=1,label={RQ\arabic*.}]
	\item How many hours do you see and/or read about robots on average per week (include robots on TV)?
	\item If you work with robots currently, how many hours do you work with robots (or on robotics related topics) per week?
	\item If you have worked with robots, but do not work anymore, how many hours have you worked on them?
	\item How will you rate your knowledge of robots?
\end{enumerate}

For each question, the participant had to answer in hours and chose between `0', `less than 5', `5-10' `10-15', `15-20', `20-25', `25-30', `more than 30'.


%\subsection{Statistical correction}
%As reported earlier, every participant in our study participated in three conditions: the robot co-worker condition, and two of the remaining conditions. We thus compare the behavior of the participant in any condition with the robot co-worker. For each participant, there were thus two comparisons made. Correspondingly, in our comparisons in Fig.~\ref{fig:slope_allcond}, we use a Bonferroni correction of (3 conditions - 1) 2, and all p values below 0.05 were multiplied by 2. Therefore, note that all the comparisons between conditions in Fig.~\ref{fig:slope_allcond} are between equal number of participants (and we use a one sample T-test).

%%%%%%%%%%%%%%%%%%%%%%%%%%%%%%%%%%%%%%%%%%%%%%%%%%%%%%%%%%%%%%%%%%%%%%%%%%%%%%%%

%\clearpage

\section{Results}

\subsection{Robot behavior influences human movement frequency}

Fig.~\ref{fig:allfit} shows the change of participant's {\it htp} (the average time between two consecutive alternate touches) between the together period and alone-period ($\tau_p^t(i)-\tau_p^a(i)$), relative to the {\it htp} of the co-worker behavior in the same trial ($\tau_c (i)-Av(\tau_p^a)$), where $Av(\tau_p^a)$ represents the average undisturbed {\it htp} by a participants across all his/her participant-alone periods. Note that the {\it htp} of both co-workers (robot and human) was random across trials, and the data of the participant behaviors are grouped and arranged in increasing order of co-worker's {\it htp} on the abscissa in Fig.~\ref{fig:allfit}. Later the slope of the polynomial at the lowest data abscissa was collected as a measure of how the participant {\it htp} was affected by the co-worker {\it htp}.
We found that, the slope distribution was not normal across the participants in the robot co-worker condition (p$<$0.05, Shapiro-Wilk test, median=0.017) and distribution was significantly positive across participants (median=0.017, Z(38)=3.70, p=0.0002, Signed Rank test). The robot performance {\it htp} (hence frequency) has influenced the human participants, as can be seen by the positive slopes (light blue data) in Fig.~\ref{fig:slope_allcond}. Primarily, the longer \textit{htp} of robot has caused the human participant's {\it htp} to increase (see first quadrants of Fig.~\ref{fig:allfit}A) but for many participants, this increase had a threshold after which the participant's {\it htp} decreased. Therefore because of this behavior, we found a second order fit to explain the data better with many participants using AIC. On the other hand, the shorter \textit{htp} of robot (only in robot co-worker condition) has also caused the participant's {\it htp} to decrease (3rd quadrants of Fig.~\ref{fig:allfit}A), emphasizing that a faster robot made the participants frequency higher. Not surprisingly, the {\it htp} results were similar with the human co-worker. The positive slopes (orange data) in Fig.~\ref{fig:slope_allcond} illustrates that the human participants performance (median=0.012, p=0.0017, Signed Rank test) was influenced by the human co-worker's performance {\it htp} (frequency).


\subsection{Press accuracy in the human not affected by robot co-worker}
 
We next analyze and measure the task performance of the human co-worker (participants) as a means of whether and how their touch accuracy has changed alongside the contagions in their \textit{htp}. It is important to remember that the provided target circles to the participants were large enough in diameter (5~cm diameter) and the participants were free of constraints as to where inside the target circle they should be touching. Hence the position of participant's touches were susceptible to change and could cause increase in variance (touch positions) due to their movement speed, without violating the task. Although, interestingly we found that while the participants movement frequency or {\it htp} has changed, but we didn't find any trend in the participants touch accuracy during the task. 

Across all participants and when they worked alone in the participant-alone period, we found the mean touch positions were ($\overline{X}$=0.95~cm; $\overline{Y}$=1.17~cm) and the mean standard deviations were (std(X)=0.23~cm; std(Y)=0.79~cm) from the center of the circle. It is very crucial to note that the participants were also able to maintain the same touch positions (there was no change of mean touch positions $\overline{X}$: p=0.64; $\overline{Y}$: p=0.86) and mean standard deviations (change of std(X): p=0.56; std(Y): p=0.41) between when they worked alone and when they worked with the robot co-worker (Fig.~\ref{fig:variance}A), suggesting that the robot did not affect participants task accuracies.

Similarly in the human co-worker condition in which the participants worked with another (unfamiliar) human experimenter, we found constant press accuracy, with no observed changes in the mean touch positions ($\overline{X}$: p=0.69; $\overline{Y}$: p=0.83; Fig.~\ref{fig:variance}B) and mean standard deviations (std(X): p=0.56; std(Y): p=0.39; Fig.~\ref{fig:variance}B). Remember that the provided target circles to the participants were large enough in diameter (5~cm diameter) and the participants were free of constraints as to where inside the target circle they should be touching. Although the participants could have changed their touch position and variance while still satisfying the required task, however but we do not observe this trend. 

In this experiment, together, the lack of change in task accuracy alongside the change in movement frequency, shows that robot as well as human co-workers were able to influence the participant task performance.


\subsection{Human form matters}

Interestingly, in the robot covered co-worker condition and human covered co-worker condition, covering the head and torso eliminated the contagions in the participant's \textit{htp}. The participant's {\it htp}s were no longer affected in the robot covered co-worker (T(16)=-0.3, p=0.78; dark blue data in Fig.~\ref{fig:slope_allcond}) and the human covered co-worker (T(12)=0.24, p=0.82; dark orange data in Fig.~\ref{fig:slope_allcond}). These effects were significantly lower than the effects induced in the same participants in the robot co-worker condition (T(16)=2.74, p=0.028, Bonferroni corrected, one sample T-test between robot co-worker and robot covered co-worker; T(12)=2.50, p=0.054, Bonferroni corrected, one sample T-test between robot co-worker and human covered co-worker). These results emphasize that the human form is crucial for induction of the performance changes.

Finally, previous studies have shown that motor contagions are not present when the robot movements are not biologically inspired or non-biological (industrial) in nature~\cite{Kilner:CurBio:2003,Bisio:PlosOne:2014}, therefore we added another two conditions (robot non-biol and robot indus) with robot co-worker (see subsection~\nameref{HRP2_Traj}), in which the participants could see the robot whole upper body, but in robot indus condition, robot movements were inspired from the traditional trapezoidal shape velocity profile trajectory while in the robot non-biological condition, robot movements were designed by gaining insights from both the biological and industrial trajectories. Agreeing with previous studies, we did not find any significant change in {\it htp}s in this condition (Z(16)=-1.07, p=0.29 (cyan data); Z(17)=-0.28, p=0.77 (magenta data) in Fig.~\ref{fig:slope_allcond}), and these values were different (tending to significance) compared to the robot co-worker condition of the same participants (robot non-biol condition: T(16)=2.32, p=0.066, Bonferroni corrected, one sample T-test) and (robot indus condition: T(17)=2.53, p=0.043, Bonferroni corrected, one sample T-test).


\subsection{The performance effect were implicit} \label{questionnaire}

To further analyze the post experiment questionnaire across the participants, we took an average of the scores from Q1, Q2, Q5 and Q4 (value subtracted from 7) and found the score to be equal to (mean$\pm$SD, 1.90$\pm$0.18) for the robot co-worker condition, and (mean$\pm$SD, 1.65$\pm$0.24) for the human co-worker condition respectively. The participants did not consciously realize the effects on their behavior as suggested by these low scores. The participants were not tired during the task in each condition, as confirmed by the Q3, the obtained scores of (mean$\pm$SD, 0.96$\pm$0.18) across the participants in the robot co-worker condition, and (mean$\pm$SD, 0.75$\pm$0.22) in the human co-worker condition. Finally, the participants did not hear any external audio cues from either the robot's joints in the robot co-worker conditions (mean$\pm$SD, 0.5$\pm$1.25), nor the human co-worker's touches in the human co-worker conditions (mean$\pm$SD, 0.58$\pm$1.36) as confirmed by Q6.


\begin{figure}[hptb]
	\centering{\includegraphics[width=0.7\columnwidth]{plots/c3-plots/Fig8}}
	\caption{{\bf Robot experience exposure.} The plot of the change of participant {\it htp}, with respect to their prior robot exposure and experience (self-scored by participants) showed a significant correlation between the two.}
	\label{fig:corelation}
\end{figure}


\subsubsection{Contagion increases with robot exposure}

During the 2$^{nd}$ questionnaire, which we had sent out at the end of our experiment and data collection. 23 participants had answered on our robot exposure questionnaire. One participant out of these 23 who rated scored `0' for all questions was removed. We then took the average scores (taking either one from RQ2 and RQ3, as they were complementary) for the others and plotted it against their {\it htp} slope in the robot co-worker condition in Fig.~\ref{fig:corelation}. Interestingly, we found a significant positive correlation (Pearson's R=0.44, p=0.039) effect, suggesting that the strength of this effect on the participants is directly proportional to more exposure and experience with robots. This result also agree with a recent report where participants with more experience with robots show higher adaptation to it, see~\cite{vannucci:roman:2017}.

%\clearpage

\section{Discussion}

To summarize the study mentioned in this Chapter, primarily, our findings suggest that the presence of a humanoid robot co-worker (or a human co-worker) can influence the performance frequencies of human participants. We observed that participants become slower with a slower co-worker, but also faster with faster co-workers. In this study, we focused on measuring and then examining the change of participant behavior `relative' to the robot behavior. Therefore, to quantify the motor contagion, we examine the ratio and analyzed the change in participant's \textit{htp} between the together period and alone-period in a trial and relative to the {\it htp} of the co-worker behavior in the same trial. The negative numerator term ($\tau_p^t(i)-\tau_p^a(i)$) suggests that participants get faster from their initial participant-alone period \textit{htp} (movement speed) and vice versa. It is important to note that the removal (subtraction) in the denominator of $Av(\tau_p^a)$ is a constant, which only shifts the curve in the negative abscissa and does not effect the slope.  


Note that the results we obtain here are specific to cyclic or repetitive tasks. Since we found that robots are generally employed in industries where typical \textit{pick-n-place} tasks are the most common. Although, it has been shown that cyclic and discrete tasks may be very different in terms of neural processes~\cite{Schaal:Nature:2004}, and further studies are required to verify whether the effects that we observe here are also valid for discrete movements. Further studies are also required to understand whether and how the contagions we observed here are related to \textit{Motor entertainment}, which is a phenomena predominantly defined for rhythmic auditory stimuli~\cite{Tierney:Frontiers:2014,Schachner:Elsevier:2009}. In our task we provided participants with white noise feedback to avoid hearing noise or audio cues from the moving robot, having said that, it may be possible the effect we observed here could be type of \textit{visual} Motor entertainment. Though at the moment, nothing can be said more concretely on this topic.


Our results show that the human and humanoid robot co-workers have been able to affect the performance frequencies of the participants, while their task accuracy (touch press) remained undisturbed and unaffected (Fig.~\ref{fig:variance}). However with the robot co-worker, this is true only when the robot head and torso were visible and robot made biological movements. A slower robot co-worker was able to reduce human performance (in terms of speed and accuracy), while a faster robot co-worker improves it. Recent studies have pointed out that some specialized robots can have an impact on both human performance and motivation during physical~\cite{Takagi:Nature:2017} and cognitive~\cite{Fasola:ICDL:2010} interactions. However here in our results, we show that the mere presence of humanoid robots can instigate effects in human performance. 


Interestingly, among all six conditions that we examined, we only observed the effect on the movement frequency when the head and torso of the co-worker (both human and robot) was visible to the participants (Fig.~\ref{fig:slope_allcond}), mainly in human co-worker condition and robot co-worker condition, suggesting that the human form plays a determining role for these effects over the biological nature of the movements made. Instead of using a traditional robotic manipulator,  we primarily investigated the effect of human form by covering the head and torso of the humanoid robot due to two main reasons. Firstly, this allowed us to test the identical physical appearance of the robotic arm and its movement between the conditions robot co-worker and robot covered co-worker. Therefore we ruled out the possibility of not having this effect in robot covered co-worker condition is merely because of two different robots were used. Secondly this also allowed us to clarify that induction of motor contagions and its effect are not influenced by the \textit{presence of a humanoid co-worker} or that participants were aware of it, but rather by the visibility of head and torso of the humanoid robot co-worker. By using a traditional manipulator, both of these issues would had been remained unclear. However, these results gave us and others new opportunity for several new questions to be researched in near future. First, we now know that the visibility of the head and torso of a humanoid modulates the motor contagions in the human co-worker but the reasons behind this are still unclear. We could argue that this effect is probably related to aspects of saliency as the torso not only occupies a larger visual field, but (especially the head and the eyes) also probably attracts participant attention when present. Second, we examined the conditions when the head and torso remained static in our task while the robot co-worker made predominantly arm movements. At this point, it remains to clarify how the torso movements would affect contagions. While we believe, if the torso moves in a task, then the effect of the torso's visibility should increase as well. Finally, here both the participants and co-workers (human and humanoid robot HRP-2Kai) performed the same task and we analyzed their behavior in it, although this would be very interesting to examine how the contagions manifest in settings where the co-workers and participants work on different tasks, including non-industrial task that are explicitly collaborative, or competitive.


Quantitatively speaking, our observed trends were highly significant but not substantial. However, these trends were noticeable to increase within our participants, especially with their participant's robot experience (Fig.~\ref{fig:corelation}). This suggests that they can prevail over a long time and are thus may important in scenarios involving long time robot-human interactions. Note that we quantify the participants level of the robot exposure by a self perceived robot exposure score by themselves rather than the actual robot exposure. As an actual standard robot exposure questionnaire to measure this effect is currently absent and the development of one can be useful to understand how this effects, such as the one we highlight here, vary over time. Overall the findings mentioned and discussed in this Chapter highlight several new features of motor contagions, but also opens new questions for future research.


In order to moderate or exploit the contagions induced by the humanoid robot co-worker, one can find these results useful for customizing the design of robot co-workers in industries and sports in future studies. Motor contagions related to body postures or undesirable competitions could affect worker health and psychology in prolong duration, but may be minimized by controlling the physical appearance and/or kinematics of robot co-workers, while where ethically valid, contagions may also be used to improve worker performance speed and hence productivity.


\clearpage






