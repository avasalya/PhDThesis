% ====================================================================== %
%                           Introduction
% ======================================================================= %

{\color{blue}\chapter*{Introduction}}
\addcontentsline{toc}{chapter}{Introduction}
\pagestyle{plain}

Humanoid robots are an amazing yet complex systems, therefore during an human robot interaction, it is crucial to understand what makes a humanoid robot a human-like and the sense of trust that comes with it. In today's industry there is a strong need of collaboration between humans and robots. These robots are required to move around the human co-worker, interact with them, understand their need and collaborate with them if need be. The sense of safety is the key here during these interactions between human and robot co-workers. The main question that needs to be answered here is how can we make robot more human-like? whether bringing changes in their appearance or improving their control algorithms would enable them to act more human-like during an interaction with human co-worker?

This thesis contributes in the broad field of human robot interactions (specially with the humanoid robot) both at a safer distance and in close proximity during a physical interaction, namely human-robot interaction(HRI) and physical human robot interaction (pHRI). The work done in this thesis is about the interactions between human and humanoid robot as co-workers in the industrial scenarios. By interactions, we started with the non-physical human-robot interaction scenario based on a industrially inspired \textit{Pick-n-Place} task example and then advanced towards the physical human-robot interactions with an example of human humanoid robot dual arm bi-directional object handover. This thesis partially focuses on the behavioral effects of those human skills learned humanoid robots on their human co-worker during a common task and rest of thesis deals with study of physical interaction between human and robot during an interactive task of object handover.

This thesis is divided into two parts. In the context of \textit{non-physical} human-robot interactions, the studies conducted in the 1$^{st}$ part of this thesis are mostly motivated by social interactions between human and humanoid robot co-workers, which deal with the behavioral and cognitive aspects of interactions. While in the context of \textit{physical} human-robot interactions, the 2$^{nd}$ part of this thesis is motivated by the physical manipulations of object between human and humanoid robot co-workers in close proximity using humanoid robot whole-body control framework.


\textbf{Thesis Outline :} During an empirical industrial co-worker setting, in one HRI study (Chapter \ref{distinct motor contagion}), we examine the effect of motor contagions induced in participants during and after by the observation of the same movements performed by a human, or a humanoid robot co-worker. While in  another study (Chapter \ref{more than just co-workers}) we systematically varied the robot behavior, and observed whether and how the performance of a human participant is affected by the presence of the humanoid robot. We also investigated the effect of physical form of humanoid robot co-worker where torso and head were covered, and only the moving arm was visible to the human participants. Finally, we compared these behaviors with a human co-worker, and examined how the observed behavioral affects scale with experience of robots. In final pHRI study (Chapter \ref{handover chapter}) we designed intuitive bi-directional object handover routine between human and biped humanoid robot co-worker using whole-body control and locomotion, we also predict and estimate the handover position in advance along with the relative orientation of object or human hand during handover and examined the interaction forces during the handover of an unknown mass object along with the timing of object handover routine.

Next in Chapter~\ref{sota}, we will start with the quick review of previous works in the related fields.



\clearpage % end of Introduction
\pagestyle{fancy}



%%as well as the effect of robot's physical form on the timing of object handover routine.
%
%
%\textbf{some Handover thesis: \textit{check comment}}
%%\textbf{file:///C:/Users/Ashesh/Dropbox%20(CNRS-AIST%20JRL)/PhD/Handover/Literatures/already%20read/a%20%20human-inspired%20controller%20for%20%20robot-human%20object%20handovers.pdf}
%
%\textbf{start story with the effect of appearance/uncanny valley etc}
%
%\textbf{talk about HRI challenges in comanoids}
%
%
%\textit{ demonstrate how the basic principles of intuitive handover-- on a complex platform biped humanoid robot while exploiting all the capabilities of such platforms and (iii), highlight the limitations of the passivity-based approaches often used in pHRI, and thereby justify further research in the field of pHRI\\}
%
%
%
%don: start story with a focus on \textit{\textbf{humanoid robot} This definition also implies that humanoids are at the convergence of several robotics research areas: legged locomotion, manipulation control for the arms and hands, sensor processing, just to name a few} some figures of humanoids.... 
%
%don: \textit{The diversity is evident but the similarity in form allows us to be generic in our approach to programming humanoids such as these, which is one of the main goals of this thesis. Although these prototypes are certainly impressive, current day humanoids are still
%	a long way from achieving the functionalities that we desire or envision. Since the
%	form is human-like, the performance of tasks is similarly expected to be humanlike. }
%
%don: \textit{However, in recent years, there has been a significant emerging branch
%	of industrial robots that brought along a paradigm shift: collaborative robots that
%	are designed to safely work side-by-side with humans. }
%
%
%don: \textbf{This thesis is a contribution towards the convergence of the last two areas: humanoid robots with the ability of working together with humans.}
%
%don: \textbf{Humanoid robots are robots with an anthropomorphic form. However, this term is often used loosely, since the robot can have varying degrees of similarity with a
%	human. For this thesis, a humanoid is defined as a robot having two legs and two arms, that are attached to a torso with a head, all of which are functional.}
%
%
%
%vincent: \textbf{Recently, humanoid robots have been considered (i) for rescue and intervention in disaster situations Kakiuchi et al. [2017]; (ii) as home service companions to assist frail and aging people; and (iii) as collaborative workers (i.e. as cobots termed“comanoids”) in large-manufacturing assembly plants2 where wheeled and rail-ported robots cannot be used (e.g. aircrafts and shipyards), among other applications. These three example applications have different social and economic impacts, different business models, but they also have different requirements in terms of hardware, perception capabilities, and dexterity. Humanoids are highly complex systems and stabilizing them is still a big challenge.
%}
%
%vincent: \textbf{ontrol is undermined.
%	As for any other robotic systems, humanoid robots shall preserve first the human
%	integrity, secondly their surrounding environment integrity, and finally their own integrity. This is a well-know behavior formulated in the iconic so-called Asimov’s laws:
%	1. ’Law 1: A robot may not injure a human being or, through inaction, allow a
%	human being to come to harm’
%	2. ’Law 2: A robot must obey the orders given it by human beings except where
%	such orders would conflict with the First Law’
%	3. ’Law 3: A robot must protect its own existence as long as such protection does
%	not conflict with the First or Second Law’
%}
%
%
%evard: \textbf{	To avoid conflicts among the partners’ intentions, the leader-follower model
%	defines a task leader, who imposes a task plan to the other partners, while the latter
%	act as follower and follow at best the intentions of the leader. This model has often
%	been used in physical Human-Robot Interaction (pHRI). Because robotic systems
%	have limited cognitive capabilities in comparison to human beings, a follower role
%	has generally been assigned to robotic systems to cooperate with human operators.
%	Recently, thanks to the increasing computational power embedded into the robots,
%	more and more initiative has been given to robotic assistants. In some recent works,
%	robots were sometimes even given the possibility to lead human operators}
%
%
%
%bussy: \textbf{But safe interaction is not the only issue of physical human-robot interaction (pHRI). When two persons interact physically, they perceive intentions from each other. Getting
%	back to the example of transporting a piece of furniture: human dyads are able to guess each
%	other’s intentions to move the object in a certain direction and at a certain velocity. They
%	are able to negotiate the direction and velocity, without being a burden to each other and
%	even without much of voice communication: they are proactive to each other. Guessing each
%	other’s intention and proactivity are key features in human-human interaction. Therefore,
%	robots for human partnership must be endowed with the capability to guess/understand human’s intentions instantly for a variety of tasks that requires physical interaction, such as
%	carrying an object together, walking hand in hand or handshaking.\\} 

