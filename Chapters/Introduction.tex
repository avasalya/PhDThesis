% ====================================================================== %
%                           Introduction
% ======================================================================= %

{\color{blue}\chapter*{Introduction}}
\addcontentsline{toc}{chapter}{Introduction}
\pagestyle{plain}

Humanoid robots are amazing yet complex systems, therefore during a human robot interaction, it is crucial to understand what makes a humanoid robot human-like and the sense of trust that comes with it. In today's industry there is a strong need of collaboration between humans and robots. These robots are required to move around the human co-worker, interact with them, understand their need and collaborate with them if need be. The sense of safety is the key here during these interactions between human and robot co-workers. The main question that needs to be answered here is how can we make robot more human-like? whether bringing changes in their appearance or improving their control algorithms would enable them to act more human-like during an interaction with human co-worker?

This thesis contributes in the broad field of human robot interactions (specially with the humanoid robot) both at a safer distance and in close proximity, namely during a human-robot interaction (HRI) and physical human robot interaction (pHRI) respectively. The work done in this thesis is about the interactions between human and humanoid robot as co-workers in the industrial scenarios. By interactions, we started with the non-physical human-robot interaction scenario based on an industrially inspired \textit{Pick-n-Place} task example and then advanced towards the physical human-robot interactions with an example of human humanoid robot dual arm bi-directional object handover. This thesis partially focuses on the behavioral effects of those human skills learned humanoid robots on their human co-worker during a common task and rest of the thesis deals with the study of physical interaction between human and robot during an interactive task of object handover.

This thesis is divided into two parts. In the context of \textit{non-physical} human-robot interactions, the studies conducted in the 1$^{st}$ part of this thesis are mostly motivated by social interactions between human and humanoid robot co-workers, which deal with the implicit behavioral and cognitive aspects of interactions. While in the context of \textit{physical} human-robot interactions, the 2$^{nd}$ part of this thesis is motivated by the physical manipulations of object between human and humanoid robot co-workers in close proximity using humanoid robot whole-body control framework.


\textbf{Thesis Outline :} In an empirical industrial co-worker setting, in one HRI study (Chapter \ref{distinct motor contagion}), we examine the effect of motor contagions induced in participants during and after the observation of the same movements performed by a human, or a humanoid robot co-worker. While in (Chapter \ref{more than just co-workers}) we systematically varied the robot behavior, and observed whether and how the performance of a human participant is affected by the presence of the humanoid robot. We also investigated the effect of physical form of humanoid robot co-worker where torso and head were covered, and only the moving arm was visible to the human participants. Later, we compared these behaviors with a human co-worker, and examined how the observed behavioral affects scale with experience of robots. Finally in the pHRI study (Chapter \ref{handover chapter}) we designed an intuitive bi-directional object handover routine between a human and a biped humanoid robot co-worker using whole-body control and locomotion. We predict and estimate the handover position and relative orientation of object or human hand during an handover and examined the interaction forces during the handover of an unknown mass object along with the timing of object handover routine.

Next in Chapter~\ref{sota}, we will start with the in-depth review of previous works in the related fields.



\clearpage % end of Introduction
\pagestyle{fancy}