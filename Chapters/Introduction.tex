% ====================================================================== %
%                           Introduction
% ======================================================================= %

{\color{blue}\chapter*{Introduction}}
\addcontentsline{toc}{chapter}{Introduction}
\pagestyle{plain}

Humanoid robots are amazing yet complex systems; therefore, during human-robot interaction, it is crucial to understand what makes the behaviour of humanoid robot human-alike and the sense of trust that comes with it. In today's industry, there is a strong need for collaboration between humans and robots. These robots are required to move around the human co-worker, interact with them, understand their need and collaborate with them if need be either in a close shared workspace or in a large cluttered environment. The core value behind these human-robot interactions is the \textit{safety} of both human and robot co-workers. As suggested by the author \textit{Isaac Asimov} in his `Three Laws of Robotics'.


\begin{enumerate}
	\item ``Law  1: A  robot  may  not  injure  a  human  being  or, through  inaction,  allow  a human being to come to harm.''
	\item ``Law 2:  A robot must obey the orders given it by human beings except where such orders would conflict with the First Law.''
	\item ``Law 3:  A robot must protect its own existence as long as such protection does not conflict with the First or Second Law.''
\end{enumerate}

The question that we try to answer here is what kind of behavioural and algorithmic improvements that we could bring for the better acceptance of a humanoid robot in industrial scenarios as co-workers? Whether bringing changes in their appearance or improving their control algorithms would enable them to act more human-alike during an interaction with a human co-worker?


This thesis contributes in the broad field of human-robot interactions (especially with the humanoid robot) both at a safer distance and nearby, namely during a human-robot interaction (HRI) and physical human-robot interaction (pHRI) respectively. The work done in this thesis is about the interactions between human and humanoid robot as co-workers in the industrial scenarios. By interactions, we started with the non-physical human-robot interaction scenario based on an industrially inspired \textit{Pick-n-Place} task example and then advanced towards the physical human-robot interactions with an example of human-humanoid robot dual-arm bi-directional object handover. 


This thesis has two parts. In the context of \textit{non-physical} human-robot interactions, the studies conducted in the 1$^\text{st}$ part of this thesis are inspired by social interactions between human and humanoid robot co-workers, which deal with the implicit behavioural and cognitive aspects of interactions. While in the context of \textit{physical} human-robot interactions, the 2$^\text{nd}$ part of this thesis is inspired by the physical manipulations and handover of the object between human and humanoid robot co-workers nearby using robot whole-body control and locomotion.


\textbf{Thesis Outline :} In an empirical industrial co-worker setting, in one HRI study (Chapter \ref{distinct motor contagion}), we examine the effect of motor contagions induced in participants during and after the observation of the same movements performed by a human, or a humanoid robot co-worker. While in (Chapter \ref{more than just co-workers}), we systematically varied the robot behavior and observed whether and how the performance of a human participant is affected by the presence of the humanoid robot. We also investigated the effect of the physical form of humanoid robot co-worker where the torso and head were covered, and only the moving arm was visible to the human participants. Later, we compared these behaviours with a human co-worker and examined how the observed behavioural effects scale with experience of robots. Finally, in the pHRI study (Chapter \ref{handover chapter}), we designed an intuitive bi-directional object handover framework between a human and a biped humanoid robot co-worker using whole-body control and locomotion. We predicted and estimated the handover position and relative orientation of an object or human hand during a handover and examined the interaction forces during the handover of an unknown mass object along with the overall duration of object handover routine.

Next in Chapter~\ref{sota}, we start with the in-depth review of previous works in the related fields.



\clearpage % end of Introduction
\pagestyle{fancy}