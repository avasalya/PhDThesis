% ======================================================================= %
%                           Introduction
% ======================================================================= %
{\color{blue}\chapter*{Introduction}}
\addcontentsline{toc}{chapter}{Introduction}

Humanoid robots are an amazing yet complex system, therefore during an human robot interaction, it is crucial to understand what makes a humanoid robot a human-like and the sense of trust that comes with it.

\textbf{explain term co-worker}

\textbf{start story with the effect of appearance/uncanny valley etc}

\textbf{talk about HRI challenges in comanoids}

\textbf{These three example applications have different social and economic impacts, different business models, but they also have different requirements in terms of hardware, perception capabilities, and dexterity. Humanoids are highly complex systems and stabilizing them is still a big challenge.
}

In today's industry there is a strong need of collaboration between human and robot, these robots are required to move around the human co-worker, interact with them, understand their need and communicate with them if need be. The sense of safety is the key here during these interactions between human and robot co-worker. The main question that needs to be answered here is the sense of understanding during an interaction between human and robot co-worker.

%\addcontentsline{toc}{section}{Research Goals}
\paragraph*{Research Goals\\}

%\addcontentsline{toc}{section}{Contributions}
\paragraph*{Contributions\\}

This thesis contributes in the broad field of human robot interactions (specially with the humanoid robot) both at a safer distance and physical interaction, namely human-robot interaction(HRI) and physical human robot interaction(pHRI). The work done in this thesis is about the interactions between human and humanoid robot as co-workers in the industrial scenarios. By interactions, we started with a non-physical human-robot interaction scenario based on a industrially inspired \textit{Pick-n-Place} task example and then try to move towards physical human-robot interactions with an example of human humanoid dual arm bi-directional object handover.


%\addcontentsline{toc}{section}{Thesis outline}
\paragraph*{Thesis Outline\\}
During an empirical industrial co-worker setting, in one HRI study (Chapter \ref{distinct motor contagion}), we examine the effect of motor contagions induced in participants during and after by the observation of the same movements performed by a human, or a humanoid robot co-worker. While in  another study (Chapter \ref{more than just co-workers}) we systematically varied the robot behavior, and observed whether and how the performance of a human participant is affected by the presence of the humanoid robot. We also investigated the effect of physical form of humanoid robot co-worker where torso and head were covered, and only the moving arm was visible to the human participants. Finally, we compared these behaviors with a human co-worker, and examined how the observed behavioral affects scale with experience of robots. In final pHRI study (Chapter \ref{handover chapter}) we designed intuitive bi-directional object handover routine between human and humanoid robot using whole-body control and locomotion, we also predict and estimate the handover position in advance along with the relative orientation of object or human hand during handover and examined the interaction forces during handover of an unknown mass object as well as the effect of robot's physical form on the timing of object handover routine.





%\paragraph*{Chapter \ref{sota}}
%talk little about each chapter
%
%\paragraph*{Chapter \ref{distinct motor contagion}}
%
%\paragraph*{Chapter \ref{more than just co-workers}}

%
%\paragraph*{Chapter \ref{handover chapter}}


\clearpage
